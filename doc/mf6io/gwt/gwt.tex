This section describes the data files for a \mf Groundwater Transport (GWT) Model.  A GWT Model is added to the simulation by including a GWT entry in the MODELS block of the simulation name file.

There are three types of spatial discretization approaches that can be used with the GWT Model: DIS, DISV, and DISU.  The input instructions for these three packages are not described here in this section on GWT Model input; input instructions for these three packages are described in the section on GWF Model input.

The GWT Model is designed to permit input to be gathered, as it is needed, from many different files.  Likewise, results from the model calculations can be written to a number of output files. The GWT Model Listing File is a key file to which the GWT model output is written.  As \mf runs, information about the GWT Model is written to the GWT Model Listing File, including much of the input data (as a record of the simulation) and calculated results.  Details about the files used by each package are provided in this section on the GWT Model Instructions.

The GWT Model reads a file called the Name File, which specifies most of the files that will be used in a simulation. Several files are always required whereas other files are optional depending on the simulation. The Output Control Package receives instructions from the user to control the amount and frequency of output.  Details about the Name File and the Output Control Package are described in this section.

\subsection{Multi-Species Transport Simulation}
TODO: Multi-species transport not implemented yet.  Only a single species can be represented at the moment.

\subsection{Units of Length and Time}
The GWF Model formulates the groundwater flow equation without using prescribed length and time units. Any consistent units of length and time can be used when specifying the input data for a simulation. This capability gives a certain amount of freedom to the user, but care must be exercised to avoid mixing units.  The program cannot detect the use of inconsistent units.

\subsection{Steady-State Simulations}
A steady-state transport simulation is represented by a single stress period having a single time step with the storage term set to zero. Setting the number and length of stress periods and time steps is the responsibility of the Timing Module of the \mf framework.

TODO: Need to implement mixed steady state and transient periods

\subsection{Solute Mass Budget}
A summary of all inflow (sources) and outflow (sinks) of solute mass is called a mass budget.  \mf calculates a mass budget for the overall model as a check on the acceptability of the solution, and to provide a summary of the sources and sinks of mass to the flow system.  The solute mass budget is printed to the GWT Model Listing File for selected time steps.

\subsection{Time Stepping}


\newpage
\subsection{GWT Model Name File}
The GWT Model Name File specifies the options and packages that are active for a GWT model.  The Name File contains two blocks: OPTIONS  and PACKAGES. The length of each line must be 299 characters or less. The lines in each block can be in any order.  Files listed in the PACKAGES block must exist when the program starts. 

Comment lines are indicated when the first character in a line is one of the valid comment characters.  Commented lines can be located anywhere in the file. Any text characters can follow the comment character. Comment lines have no effect on the simulation; their purpose is to allow users to provide documentation about a particular simulation. 

\vspace{5mm}
\subsubsection{Structure of Blocks}
\lstinputlisting[style=blockdefinition]{./mf6ivar/tex/gwt-nam-options.dat}
\lstinputlisting[style=blockdefinition]{./mf6ivar/tex/gwt-nam-packages.dat}

\vspace{5mm}
\subsubsection{Explanation of Variables}
\begin{description}
\input{./mf6ivar/tex/gwt-nam-desc.tex}
\end{description}

\begin{table}[H]
\caption{Ftype values described in this report.  The \texttt{Pname} column indicates whether or not a package name can be provided in the name file}
\small
\begin{center}
\begin{tabular*}{\columnwidth}{l l l}
\hline
\hline
Ftype & Input File Description & \texttt{Pname}\\
\hline
DIS6 & Rectilinear Discretization Input File \\
DISV6 & Discretization by Vertices Input File \\
DISU6 & Unstructured Discretization Input File \\
IC6 & Initial Conditions Package \\
OC6 & Output Control Option \\
ADV6 & Advection Package \\ 
DSP6 & Dispersion Package \\ 
STO6 & Storage Package \\
RCT6 & Reactions Package \\
IMD6 & Immobile Domain Package \\
SSM6 & Source and Sink Mixing Package \\ 
CNC6 & Constant Concentration Package & *\\ 
SRC6 & Mass Source Loading Package & * \\ 
OBS6 & Observations Option \\
\hline 
\end{tabular*}
\label{table:ftype}
\end{center}
\normalsize
\end{table}

\vspace{5mm}
\subsubsection{Example Input File}
\lstinputlisting[style=inputfile]{./mf6ivar/examples/gwt-nam-example.dat}



%\newpage
%\subsection{Structured Discretization (DIS) Input File}
%\input{gwf/dis}

%\newpage
%\subsection{Discretization with Vertices (DISV) Input File}
%\input{gwf/disv}

%\newpage
%\subsection{Unstructured Discretization (DISU) Input File}
%Discretization information for unstructured grids is read from the file that is specified by ``DISU6'' as the file type.  Only one discretization input file (DISU6, DISV6 or DIS6) can be specified for a model.

The shape and position of each cell can be defined using vertices.  This information is optional and is only read if the number of vertices (NVERT) in the DIMENSIONS block is specified and is assigned a value larger than zero.  If the vertices and two-dimensional cell information is provided in this file, then this information is also written to the binary grid file.  Providing this information may be useful for other postprocessing programs that read the binary grid file.

The DISU Package does not support the concept of layers, which is different from the DISU implementation in MODFLOW-USG.  In \mf~all grid input and output for models that use the DISU Package is entered or written as a one-dimensional array of size nodes.

\vspace{5mm}
\subsubsection{Structure of Blocks}
\lstinputlisting[style=blockdefinition]{./mf6ivar/tex/gwf-disu-options.dat}
\lstinputlisting[style=blockdefinition]{./mf6ivar/tex/gwf-disu-dimensions.dat}
\lstinputlisting[style=blockdefinition]{./mf6ivar/tex/gwf-disu-griddata.dat}
\lstinputlisting[style=blockdefinition]{./mf6ivar/tex/gwf-disu-connectiondata.dat}
\lstinputlisting[style=blockdefinition]{./mf6ivar/tex/gwf-disu-vertices.dat}
\lstinputlisting[style=blockdefinition]{./mf6ivar/tex/gwf-disu-cell2d.dat}

\vspace{5mm}
\subsubsection{Explanation of Variables}
\begin{description}
% DO NOT MODIFY THIS FILE DIRECTLY.  IT IS CREATED BY mf6ivar.py 

\item \textbf{Block: OPTIONS}

\begin{description}
\item \texttt{length\_units}---is the length units used for this model.  Values can be ``FEET'', ``METERS'', or ``CENTIMETERS''.  If not specified, the default is ``UNKNOWN''.

\item \texttt{NOGRB}---keyword to deactivate writing of the binary grid file.

\item \texttt{xorigin}---x-position of the origin used for model grid vertices.  This value should be provided in a real-world coordinate system.  A default value of zero is assigned if not specified.  The value for XORIGIN does not affect the model simulation, but it is written to the binary grid file so that postprocessors can locate the grid in space.

\item \texttt{yorigin}---y-position of the origin used for model grid vertices.  This value should be provided in a real-world coordinate system.  If not specified, then a default value equal to zero is used.  The value for YORIGIN does not affect the model simulation, but it is written to the binary grid file so that postprocessors can locate the grid in space.

\item \texttt{angrot}---counter-clockwise rotation angle (in degrees) of the model grid coordinate system relative to a real-world coordinate system.  If not specified, then a default value of 0.0 is assigned.  The value for ANGROT does not affect the model simulation, but it is written to the binary grid file so that postprocessors can locate the grid in space.

\end{description}
\item \textbf{Block: DIMENSIONS}

\begin{description}
\item \texttt{nodes}---is the number of cells in the model grid.

\item \texttt{nja}---is the sum of the number of connections and NODES.  When calculating the total number of connections, the connection between cell n and cell m is considered to be different from the connection between cell m and cell n.  Thus, NJA is equal to the total number of connections, including n to m and m to n, and the total number of cells.

\item \texttt{nvert}---is the total number of (x, y) vertex pairs used to define the plan-view shape of each cell in the model grid.  If NVERT is not specified or is specified as zero, then the VERTICES and CELL2D blocks below are not read.  NVERT and the accompanying VERTICES and CELL2D blocks should be specified for most simulations.  If the XT3D or SAVE\_SPECIFIC\_DISCHARGE options are specified in the NPF Package, these this information is required.

\end{description}
\item \textbf{Block: GRIDDATA}

\begin{description}
\item \texttt{top}---is the top elevation for each cell in the model grid.

\item \texttt{bot}---is the bottom elevation for each cell.

\item \texttt{area}---is the cell surface area (in plan view).

\end{description}
\item \textbf{Block: CONNECTIONDATA}

\begin{description}
\item \texttt{iac}---is the number of connections (plus 1) for each cell.  The sum of all the entries in IAC must be equal to NJA.

\item \texttt{ja}---is a list of cell number (n) followed by its connecting cell numbers (m) for each of the m cells connected to cell n. The number of values to provide for cell n is IAC(n).  This list is sequentially provided for the first to the last cell. The first value in the list must be cell n itself, and the remaining cells must be listed in an increasing order (sorted from lowest number to highest).  Note that the cell and its connections are only supplied for the GWF cells and their connections to the other GWF cells.  Also note that the JA list input may be divided such that every node and its connectivity list can be on a separate line for ease in readability of the file. To further ease readability of the file, the node number of the cell whose connectivity is subsequently listed, may be expressed as a negative number, the sign of which is subsequently converted to positive by the code.

\item \texttt{ihc}---is an index array indicating the direction between node n and all of its m connections.  If IHC = 0 then cell n and cell m are connected in the vertical direction.  Cell n overlies cell m if the cell number for n is less than m; cell m overlies cell n if the cell number for m is less than n.  If IHC = 1 then cell n and cell m are connected in the horizontal direction.  If IHC = 2 then cell n and cell m are connected in the horizontal direction, and the connection is vertically staggered.  A vertically staggered connection is one in which a cell is horizontally connected to more than one cell in a horizontal connection.

\item \texttt{cl12}---is the array containing connection lengths between the center of cell n and the shared face with each adjacent m cell.

\item \texttt{hwva}---is a symmetric array of size NJA.  For horizontal connections, entries in HWVA are the horizontal width perpendicular to flow.  For vertical connections, entries in HWVA are the vertical area for flow.  Thus, values in the HWVA array contain dimensions of both length and area.  Entries in the HWVA array have a one-to-one correspondence with the connections specified in the JA array.  Likewise, there is a one-to-one correspondence between entries in the HWVA array and entries in the IHC array, which specifies the connection type (horizontal or vertical).  Entries in the HWVA array must be symmetric; the program will terminate with an error if the value for HWVA for an n to m connection does not equal the value for HWVA for the corresponding n to m connection.

\item \texttt{angldegx}---is the angle (in degrees) between the horizontal x-axis and the outward normal to the face between a cell and its connecting cells. The angle varies between zero and 360.0 degrees, where zero degrees points in the positive x-axis direction, and 90 degrees points in the positive y-axis direction.  ANGLDEGX is only needed if horizontal anisotropy is specified in the NPF Package, if the XT3D option is used in the NPF Package, or if the SAVE\_SPECIFIC\_DISCHARGE option is specifed in the NPF Package.  ANGLDEGX does not need to be specified if these conditions are not met.  ANGLDEGX is of size NJA; values specified for vertical connections and for the diagonal position are not used.  Note that ANGLDEGX is read in degrees, which is different from MODFLOW-USG, which reads a similar variable (ANGLEX) in radians.

\end{description}
\item \textbf{Block: VERTICES}

\begin{description}
\item \texttt{iv}---is the vertex number.  Records in the VERTICES block must be listed in consecutive order from 1 to NVERT.

\item \texttt{xv}---is the x-coordinate for the vertex.

\item \texttt{yv}---is the y-coordinate for the vertex.

\end{description}
\item \textbf{Block: CELL2D}

\begin{description}
\item \texttt{icell2d}---is the cell2d number.  Records in the CELL2D block must be listed in consecutive order from 1 to NODES.

\item \texttt{xc}---is the x-coordinate for the cell center.

\item \texttt{yc}---is the y-coordinate for the cell center.

\item \texttt{ncvert}---is the number of vertices required to define the cell.  There may be a different number of vertices for each cell.

\item \texttt{icvert}---is an array of integer values containing vertex numbers (in the VERTICES block) used to define the cell.  Vertices must be listed in clockwise order.

\end{description}


\end{description}

\vspace{5mm}
\subsubsection{Example Input File}
\lstinputlisting[style=inputfile]{./mf6ivar/examples/gwf-disu-example.dat}



\newpage
\subsection{Initial Conditions (IC) Package}
Initial Conditions (IC) Package information is read from the file that is specified by ``IC6'' as the file type.  Only one IC Package can be specified for a GWT model. 

\vspace{5mm}
\subsubsection{Structure of Blocks}
%\lstinputlisting[style=blockdefinition]{./mf6ivar/tex/gwf-ic-options.dat}
\lstinputlisting[style=blockdefinition]{./mf6ivar/tex/gwt-ic-griddata.dat}

\vspace{5mm}
\subsubsection{Explanation of Variables}
\begin{description}
% DO NOT MODIFY THIS FILE DIRECTLY.  IT IS CREATED BY mf6ivar.py 

\item \textbf{Block: GRIDDATA}

\begin{description}
\item \texttt{strt}---is the initial (starting) concentration---that is, concentration at the beginning of the GWT Model simulation.  STRT must be specified for all simulations, including steady-state simulations. One value is read for every model cell. For simulations in which the first stress period is steady state, the values used for STRT generally do not affect the simulation. The execution time, however, will be less if STRT includes concentrations that are close to the steady-state solution.

\end{description}


\end{description}

\vspace{5mm}
\subsubsection{Example Input File}
\lstinputlisting[style=inputfile]{./mf6ivar/examples/gwt-ic-example.dat}



\newpage
\subsection{Output Control (OC) Option}
Input to the Output Control Option of the Groundwater Transport Model is read from the file that is specified as type ``OC6'' in the Name File. If no ``OC6'' file is specified, default output control is used. The Output Control Option determines how and when concentrations are printed to the listing file and/or written to a separate binary output file.  Under the default, head and overall budget are written to the Listing File at the end of every stress period. The default printout format for concentrations is 10G11.4.

Output Control data must be specified using words.  The numeric codes supported in earlier MODFLOW versions can no longer be used.

For the PRINT and SAVE options of heads, there is no longer an option to specify individual layers.  Whenever one of these arrays is printed or saved, all layers are printed or saved.

\vspace{5mm}
\subsubsection{Structure of Blocks}
\vspace{5mm}

\noindent \textit{FOR EACH SIMULATION}
\lstinputlisting[style=blockdefinition]{./mf6ivar/tex/gwt-oc-options.dat}
\vspace{5mm}
\noindent \textit{FOR ANY STRESS PERIOD}
\lstinputlisting[style=blockdefinition]{./mf6ivar/tex/gwt-oc-period.dat}

\vspace{5mm}
\subsubsection{Explanation of Variables}
\begin{description}
\input{./mf6ivar/tex/gwt-oc-desc.tex}
\end{description}

\vspace{5mm}
\subsubsection{Example Input File}
\lstinputlisting[style=inputfile]{./mf6ivar/examples/gwt-oc-example.dat}


\newpage
\subsection{Advection (ADV) Package}
Advection (ADV) Package information is read from the file that is specified by ``ADV6'' as the file type.  Only one ADV Package can be specified for a GWT model. 

\vspace{5mm}
\subsubsection{Structure of Blocks}
\lstinputlisting[style=blockdefinition]{./mf6ivar/tex/gwt-adv-options.dat}

\vspace{5mm}
\subsubsection{Explanation of Variables}
\begin{description}
% DO NOT MODIFY THIS FILE DIRECTLY.  IT IS CREATED BY mf6ivar.py 

\item \textbf{Block: OPTIONS}

\begin{description}
\item \texttt{scheme}---scheme used to solve the advection term.  Can be upstream, central, or TVD.

\end{description}


\end{description}

\vspace{5mm}
\subsubsection{Example Input File}
\lstinputlisting[style=inputfile]{./mf6ivar/examples/gwt-adv-example.dat}



\newpage
\subsection{Dispersion (DSP) Package}
Dispersion (DSP) Package information is read from the file that is specified by ``DSP6'' as the file type.  Only one DSP Package can be specified for a GWT model. 

\vspace{5mm}
\subsubsection{Structure of Blocks}
\lstinputlisting[style=blockdefinition]{./mf6ivar/tex/gwt-dsp-options.dat}
\lstinputlisting[style=blockdefinition]{./mf6ivar/tex/gwt-dsp-griddata.dat}

\vspace{5mm}
\subsubsection{Explanation of Variables}
\begin{description}
% DO NOT MODIFY THIS FILE DIRECTLY.  IT IS CREATED BY mf6ivar.py 

\item \textbf{Block: OPTIONS}

\begin{description}
\item \texttt{XT3D}---activate the xt3d method to solve the dispersion term

\end{description}
\item \textbf{Block: GRIDDATA}

\begin{description}
\item \texttt{diffc}---molecular diffusion coefficient.

\item \texttt{alh}---longitudinal dispersivity in horizontal direction.  If mechanical dispersion is represented (by specifying any dispersivity values) then this array is required.

\item \texttt{alv}---longitudinal dispersivity in vertical direction.  If this value is not specified and mechanical dispersion is represented, then this array is set equal to ALH.

\item \texttt{ath1}---transverse dispersivity in horizontal direction.  If mechanical dispersion is represented (by specifying any dispersivity values) then this array is required.

\item \texttt{ath2}---transverse dispersivity in horizontal direction.  If this value is not specified and mechanical dispersion is represented, then this array is set equal to ATH1.

\item \texttt{atv}---transverse dispersivity when flow is in vertical direction.  If this value is not specified and mechanical dispersion is represented, then this array is set equal to ATH2.

\end{description}


\end{description}

\vspace{5mm}
\subsubsection{Example Input File}
\lstinputlisting[style=inputfile]{./mf6ivar/examples/gwt-dsp-example.dat}



\newpage
\subsection{Storage (STO) Package}
Storage (STO) Package information is read from the file that is specified by ``STO6'' as the file type.  Only one STO Package can be specified for a GWT model. 

\vspace{5mm}
\subsubsection{Structure of Blocks}
\lstinputlisting[style=blockdefinition]{./mf6ivar/tex/gwt-sto-options.dat}
\lstinputlisting[style=blockdefinition]{./mf6ivar/tex/gwt-sto-griddata.dat}

\vspace{5mm}
\subsubsection{Explanation of Variables}
\begin{description}
% DO NOT MODIFY THIS FILE DIRECTLY.  IT IS CREATED BY mf6ivar.py 

\item \textbf{Block: OPTIONS}

\begin{description}
\item \texttt{SAVE\_FLOWS}---keyword to indicate that STO flow terms will be written to the file specified with ``BUDGET FILEOUT'' in Output Control.

\end{description}
\item \textbf{Block: GRIDDATA}

\begin{description}
\item \texttt{porosity}---is the aquifer porosity.

\end{description}


\end{description}

\vspace{5mm}
\subsubsection{Example Input File}
\lstinputlisting[style=inputfile]{./mf6ivar/examples/gwt-sto-example.dat}



\newpage
\subsection{Decay (DCY) Package}
Decay (DCY) Package information is read from the file that is specified by ``DCY6'' as the file type.  Only one DCY Package can be specified for a GWT model. 

\vspace{5mm}
\subsubsection{Structure of Blocks}
\lstinputlisting[style=blockdefinition]{./mf6ivar/tex/gwt-dcy-options.dat}
\lstinputlisting[style=blockdefinition]{./mf6ivar/tex/gwt-dcy-griddata.dat}

\vspace{5mm}
\subsubsection{Explanation of Variables}
\begin{description}
% DO NOT MODIFY THIS FILE DIRECTLY.  IT IS CREATED BY mf6ivar.py 

\item \textbf{Block: OPTIONS}

\begin{description}
\item \texttt{SAVE\_FLOWS}---keyword to indicate that DCY flow terms will be written to the file specified with ``BUDGET FILEOUT'' in Output Control.

\item \texttt{ZERO\_ORDER\_DECAY}---is a text keyword to indicate that zero-order decay will occur.  If not specified the solute decay rate will be of first order.

\end{description}
\item \textbf{Block: GRIDDATA}

\begin{description}
\item \texttt{rc}---is the rate coefficient for first or zero-order decay for the aqueous phase of the mobile domain.  A negative value indicates solute production.  The dimensions of rc for a first-order reaction is one over time.  The dimensions of rc for a zero-order reaction is mass per length cubed per time.

\end{description}


\end{description}

\vspace{5mm}
\subsubsection{Example Input File}
\lstinputlisting[style=inputfile]{./mf6ivar/examples/gwt-dcy-example.dat}



\newpage
\subsection{Sorbtion (SRB) Package}
Sorbtion (SRB) Package information is read from the file that is specified by ``SRB6'' as the file type.  Only one SRB Package can be specified for a GWT model. 

\vspace{5mm}
\subsubsection{Structure of Blocks}
\lstinputlisting[style=blockdefinition]{./mf6ivar/tex/gwt-srb-options.dat}
\lstinputlisting[style=blockdefinition]{./mf6ivar/tex/gwt-srb-griddata.dat}

\vspace{5mm}
\subsubsection{Explanation of Variables}
\begin{description}
% DO NOT MODIFY THIS FILE DIRECTLY.  IT IS CREATED BY mf6ivar.py 

\item \textbf{Block: OPTIONS}

\begin{description}
\item \texttt{SAVE\_FLOWS}---keyword to indicate that SRB flow terms will be written to the file specified with ``BUDGET FILEOUT'' in Output Control.

\item \texttt{FIRST\_ORDER\_DECAY}---is a text keyword to indicate that first-order decay will occur for sorbed solute mass.  If first-order decay is specified, then the RC variable must be specified in the GRIDDATA block.

\item \texttt{ZERO\_ORDER\_DECAY}---is a text keyword to indicate that zero-order decay will occur for sorbed solute mass. If zero-order decay is specified, then the RC variable must be specified in the GRIDDATA block.

\end{description}
\item \textbf{Block: GRIDDATA}

\begin{description}
\item \texttt{rhob}---is the bulk density of the aquifer in mass per length cubed.

\item \texttt{distcoef}---is the distribution coefficient for the equilibrium-controlled linear sorption isotherm in dimensions of length cubed per mass.

\item \texttt{rc}---is the rate coefficient for first or zero-order decay for the sorbed phase.  The dimensions of rc for first-order decay is one over time.  The dimensions of rc for zero-order decay is mass of solute per mass of aquifer per time.

\end{description}


\end{description}

\vspace{5mm}
\subsubsection{Example Input File}
\lstinputlisting[style=inputfile]{./mf6ivar/examples/gwt-srb-example.dat}



\newpage
\subsection{Immobile Domain (IMD) Package}
Immobile Domain (IMD) Package information is read from the file that is specified by ``IMD6'' as the file type.  Any number of IMD Packages can be specified for a single GWT model.  This allows the user to specify triple porosity systems, or systems with as many immobile domains as necessary. 

\vspace{5mm}
\subsubsection{Structure of Blocks}
\lstinputlisting[style=blockdefinition]{./mf6ivar/tex/gwt-imd-options.dat}
\lstinputlisting[style=blockdefinition]{./mf6ivar/tex/gwt-imd-griddata.dat}

\vspace{5mm}
\subsubsection{Explanation of Variables}
\begin{description}
% DO NOT MODIFY THIS FILE DIRECTLY.  IT IS CREATED BY mf6ivar.py 

\item \textbf{Block: OPTIONS}

\begin{description}
\item \texttt{SAVE\_FLOWS}---keyword to indicate that IMD flow terms will be written to the file specified with ``BUDGET FILEOUT'' in Output Control.

\item \texttt{SORBTION}---keyword to active sorbtion

\item \texttt{decayorder}---is a text keyword to indicate that first or zero-order decay will occur.  The text value for DECAYORDER can be ``ONE'' or ``ZERO''.

\end{description}
\item \textbf{Block: GRIDDATA}

\begin{description}
\item \texttt{cim}---initial concentration of the immobile domain in mass per length cubed.  If CIM is not specified, then it is assumed to be zero.

\item \texttt{thetaim}---porosity of the immobile domain (dimensionless).

\item \texttt{zetaim}---mass transfer rate coefficient between the mobile and immobile domains, in dimenions of per time.

\item \texttt{rhob}---is the bulk density of the aquifer in mass per length cubed.  rhob will have no affect on simulation results unless the SORBTION keyword is specified in the options block.

\item \texttt{distcoef}---is the distribution coefficient for the equilibrium-controlled linear sorption isotherm in dimensions of length cubed per mass.  distcoef will have no affect on simulation results unless the SORBTION keyword is specified in the options block.

\item \texttt{rc1}---is the rate coefficient for first or zero-order decay for the aqueous phase.  A negative value indicates solute production.  The dimensions of rc1 for a first-order reaction is one over time.  The dimensions of rc1 for a zero-order reaction is mass per length cubed per time.  rc1 will have no affect on simulation results unless the DECAYORDER keyword is specified in the options block.

\item \texttt{rc2}---is the rate coefficient for first or zero-order decay for the sorbed phase.  A negative value indicates solute production.  The dimensions of rc2 for a first-order reaction is one over time.  The dimensions of rc2 for a zero-order reaction is mass of solute per mass of aquifer per time.  rc2 will have no affect on simulation results unless the SORPTION and DECAYORDER keywords are specified in the options block.

\end{description}


\end{description}

\vspace{5mm}
\subsubsection{Example Input File}
\lstinputlisting[style=inputfile]{./mf6ivar/examples/gwt-imd-example.dat}



\newpage
\subsection{Constant Concentration (CNC) Package}
Constant Concentration (CNC) Package information is read from the file that is specified by ``CNC6'' as the file type.  Any number of CNC Packages can be specified for a single GWT model, but the same cell cannot be designated as a constant concentration by more than one CNC entry. 

\vspace{5mm}
\subsubsection{Structure of Blocks}
\vspace{5mm}

\noindent \textit{FOR EACH SIMULATION}
\lstinputlisting[style=blockdefinition]{./mf6ivar/tex/gwt-cnc-options.dat}
\lstinputlisting[style=blockdefinition]{./mf6ivar/tex/gwt-cnc-dimensions.dat}
\vspace{5mm}
\noindent \textit{FOR ANY STRESS PERIOD}
\lstinputlisting[style=blockdefinition]{./mf6ivar/tex/gwt-cnc-period.dat}
\packageperioddescription

\vspace{5mm}
\subsubsection{Explanation of Variables}
\begin{description}
\input{./mf6ivar/tex/gwt-cnc-desc.tex}
\end{description}

\vspace{5mm}
\subsubsection{Example Input File}
\lstinputlisting[style=inputfile]{./mf6ivar/examples/gwt-cnc-example.dat}



\newpage
\subsection{Source and Sink Mixing (SSM) Package}
Source and Sink Mixing (SSM) Package information is read from the file that is specified by ``SSM6'' as the file type.  Only one SSM Package can be specified for a GWT model. 

\vspace{5mm}
\subsubsection{Structure of Blocks}
%\lstinputlisting[style=blockdefinition]{./mf6ivar/tex/gwt-sto-options.dat}
\lstinputlisting[style=blockdefinition]{./mf6ivar/tex/gwt-ssm-sources.dat}

\vspace{5mm}
\subsubsection{Explanation of Variables}
\begin{description}
% DO NOT MODIFY THIS FILE DIRECTLY.  IT IS CREATED BY mf6ivar.py 

\item \textbf{Block: OPTIONS}

\begin{description}
\item \texttt{PRINT\_FLOWS}---keyword to indicate that the list of SSM flow rates will be printed to the listing file for every stress period time step in which ``BUDGET PRINT'' is specified in Output Control.  If there is no Output Control option and ``PRINT\_FLOWS'' is specified, then flow rates are printed for the last time step of each stress period.

\end{description}
\item \textbf{Block: SOURCES}

\begin{description}
\item \texttt{pname}---name of the package for which an auxiliary variable contains a source concentration.

\item \texttt{component}---species number.

\item \texttt{auxname}---name of the auxiliary variable in the package PNAME that contains the source concentration for the COMPONENT species.

\end{description}


\end{description}

\vspace{5mm}
\subsubsection{Example Input File}
\lstinputlisting[style=inputfile]{./mf6ivar/examples/gwt-ssm-example.dat}



\newpage
\subsection{Mass Source Loading (SRC) Package}
Input to the Mass Source Loading (SRC) Package is read from the file that has type ``SRC6'' in the Name File.  Any number of SRC Packages can be specified for a single groundwater transport model.

\vspace{5mm}
\subsubsection{Structure of Blocks}
\vspace{5mm}

\noindent \textit{FOR EACH SIMULATION}
\lstinputlisting[style=blockdefinition]{./mf6ivar/tex/gwt-src-options.dat}
\lstinputlisting[style=blockdefinition]{./mf6ivar/tex/gwt-src-dimensions.dat}
\vspace{5mm}
\noindent \textit{FOR ANY STRESS PERIOD}
\lstinputlisting[style=blockdefinition]{./mf6ivar/tex/gwt-src-period.dat}
\packageperioddescription

\vspace{5mm}
\subsubsection{Explanation of Variables}
\begin{description}
% DO NOT MODIFY THIS FILE DIRECTLY.  IT IS CREATED BY mf6ivar.py 

\item \textbf{Block: OPTIONS}

\begin{description}
\item \texttt{auxiliary}---defines an array of one or more auxiliary variable names.  There is no limit on the number of auxiliary variables that can be provided on this line; however, lists of information provided in subsequent blocks must have a column of data for each auxiliary variable name defined here.   The number of auxiliary variables detected on this line determines the value for naux.  Comments cannot be provided anywhere on this line as they will be interpreted as auxiliary variable names.  Auxiliary variables may not be used by the package, but they will be available for use by other parts of the program.  The program will terminate with an error if auxiliary variables are specified on more than one line in the options block.

\item \texttt{auxmultname}---name of auxiliary variable to be used as multiplier of mass loading rate.

\item \texttt{BOUNDNAMES}---keyword to indicate that boundary names may be provided with the list of mass source cells.

\item \texttt{PRINT\_INPUT}---keyword to indicate that the list of mass source information will be written to the listing file immediately after it is read.

\item \texttt{PRINT\_FLOWS}---keyword to indicate that the list of mass source flow rates will be printed to the listing file for every stress period time step in which ``BUDGET PRINT'' is specified in Output Control.  If there is no Output Control option and ``PRINT\_FLOWS'' is specified, then flow rates are printed for the last time step of each stress period.

\item \texttt{SAVE\_FLOWS}---keyword to indicate that mass source flow terms will be written to the file specified with ``BUDGET FILEOUT'' in Output Control.

\item \texttt{TS6}---keyword to specify that record corresponds to a time-series file.

\item \texttt{FILEIN}---keyword to specify that an input filename is expected next.

\item \texttt{ts6\_filename}---defines a time-series file defining time series that can be used to assign time-varying values. See the ``Time-Variable Input'' section for instructions on using the time-series capability.

\item \texttt{OBS6}---keyword to specify that record corresponds to an observations file.

\item \texttt{obs6\_filename}---name of input file to define observations for the Mass Source package. See the ``Observation utility'' section for instructions for preparing observation input files. Table \ref{table:obstype} lists observation type(s) supported by the Mass Source package.

\item \texttt{MOVER}---keyword to indicate that this instance of the Mass Source Package can be used with the Water Mover (MVR) Package.  When the MOVER option is specified, additional memory is allocated within the package to store the available, provided, and received water.

\end{description}
\item \textbf{Block: DIMENSIONS}

\begin{description}
\item \texttt{maxbound}---integer value specifying the maximum number of sources cells that will be specified for use during any stress period.

\end{description}
\item \textbf{Block: PERIOD}

\begin{description}
\item \texttt{iper}---integer value specifying the starting stress period number for which the data specified in the PERIOD block apply.  IPER must be less than or equal to NPER in the TDIS Package and greater than zero.  The IPER value assigned to a stress period block must be greater than the IPER value assigned for the previous PERIOD block.  The information specified in the PERIOD block will continue to apply for all subsequent stress periods, unless the program encounters another PERIOD block.

\item \texttt{cellid}---is the cell identifier, and depends on the type of grid that is used for the simulation.  For a structured grid that uses the DIS input file, CELLID is the layer, row, and column.   For a grid that uses the DISV input file, CELLID is the layer and CELL2D number.  If the model uses the unstructured discretization (DISU) input file, CELLID is the node number for the cell.

\item \textcolor{blue}{\texttt{smassrate}---is the mass source loading rate. A positive value indicates addition of solute mass and a negative value indicates removal of solute mass. If the Options block includes a TIMESERIESFILE entry (see the ``Time-Variable Input'' section), values can be obtained from a time series by entering the time-series name in place of a numeric value.}

\item \textcolor{blue}{\texttt{aux}---represents the values of the auxiliary variables for each mass source. The values of auxiliary variables must be present for each mass source. The values must be specified in the order of the auxiliary variables specified in the OPTIONS block.  If the package supports time series and the Options block includes a TIMESERIESFILE entry (see the ``Time-Variable Input'' section), values can be obtained from a time series by entering the time-series name in place of a numeric value.}

\item \texttt{boundname}---name of the mass source cell.  BOUNDNAME is an ASCII character variable that can contain as many as 40 characters.  If BOUNDNAME contains spaces in it, then the entire name must be enclosed within single quotes.

\end{description}


\end{description}

\vspace{5mm}
\subsubsection{Example Input File}
\lstinputlisting[style=inputfile]{./mf6ivar/examples/gwt-src-example.dat}

\vspace{5mm}
\subsubsection{Available observation types}
Mass Source Loading Package observations include the simulated source loading rates (\texttt{smassrate}) and the mass source rate that is available for the MVR package (\texttt{to-mvr}). The data required for each SRC Package observation type is defined in table~\ref{table:gwt-srcobstype}. The sum of \texttt{src} and \texttt{to-mvr} is equal to the simulated mass source loading rate. Negative and positive values for an observation represent a loss from and gain to the GWT model, respectively.

\begin{longtable}{p{2cm} p{2.75cm} p{2cm} p{1.25cm} p{7cm}}
\caption{Available SRC Package observation types} \tabularnewline

\hline
\hline
\textbf{Stress Package} & \textbf{Observation type} & \textbf{ID} & \textbf{ID2} & \textbf{Description} \\
\hline
\endhead

\hline
\endfoot

SRC & src & cellid or boundname & -- & Mass source loading rate between the groundwater system and a mass source loading boundary or a group of  boundaries. \\
SRC & to-mvr & cellid or boundname & -- & Mass source loading rate that is available for the MVR package for a src boundary or a group of src boundaries.
\label{table:gwt-srcobstype}
\end{longtable}

\vspace{5mm}
\subsubsection{Example Observation Input File}
\lstinputlisting[style=inputfile]{./mf6ivar/examples/gwt-src-example-obs.dat}


%\newpage
%\subsection{Observation (OBS) Utility for a GWF Model}
%\input{gwf/gwf-obs}

