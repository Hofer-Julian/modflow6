This section describes the data files for a \mf Groundwater Flow (GWF) Model.  A GWF Model is added to the simulation by including a GWF entry in the MODELS block of the simulation name file.

There are three types of spatial discretization approaches that can be used with the GWF Model.  Input for a GWF Model may be entered in a structured form, like for previous MODFLOW versions, in that users specify cells using their layer, row, and column indices.  Users may also work with a layered grid in which cells are defined using vertices.  In this case, users specify cells using the layer number and the cell number.  Lastly, GWF Models may be entered as fully unstructured models, in which cells are specified using only their cell number.  Once a spatial discretization approach has been selected, then all input with cell indices must be entered accordingly.

The GWF Model is designed to permit input to be gathered, as it is needed, from many different files.  Likewise, results from the model calculations can be written to a number of output files. The GWF Model Listing File is a key file to which the GWF model output is written.  As \mf runs, information about the GWF Model is written to the GWF Model Listing File, including much of the input data (as a record of the simulation) and calculated results.  Details about the files used by each package are provided in this section on the GWF Model Instructions.

\mf is further designed to allow the user to control the amount, type, and frequency of information to be output. Much of the output will be written to the Simulation and GWF Model Listing Files, but some model output can be written to other files.  The Listing Files can become very large for common models.  Text editors are useful for examining the Listing File. The GWF Model Listing File includes a summary of the input data read for all packages.  In addition, the GWF Model Listing File optionally contains calculated head controlled by time step, and the overall volumetric budget controlled by time step. The Listing Files also contain information about solver convergence and error messages.  Output to other files can include head and cell-by-cell flow terms for use in calculations external to the model or in user-supplied applications such as plotting programs.

The GWF Model reads a file called the Name File, which specifies most of the files that will be used in a simulation. Several files are always required whereas other files are optional depending on the simulation. The Output Control Package receives instructions from the user to control the amount and frequency of output.  Details about the Name File and the Output Control Package are described in this section.

\subsection{Information for Existing MODFLOW Users}
\input{gwf/info_existing_users.tex}

\input{gwf/array_data.tex}

\subsection{Units of Length and Time}
The GWF Model formulates the groundwater flow equation without using prescribed length and time units. Any consistent units of length and time can be used when specifying the input data for a simulation. This capability gives a certain amount of freedom to the user, but care must be exercised to avoid mixing units.  The program cannot detect the use of inconsistent units.  For example, if hydraulic conductivity is entered in units of feet per day and pumpage as cubic meters per second, the program will run, but the results will be meaningless. Other processes generally are expected to work with consistent length and time units; however, other processes could conceivably place restrictions on which units are supported.

The user can set flags that specify the length and time units (see the input instructions for the Timing Module and Spatial Discretization Files), which may be useful in various parts of MODFLOW.  For example, the program will label the table of simulation time with time units if the time units are specified by the optional TIME\_UNITS label, which can be set in the TDIS Package.  If the time units are not specified, the program still runs, but the table of simulation time does not indicate the time units. An optional LENGTH\_UNITS label can be set in the Discretization Package. Situations in other processes may require that the length or time units be specified.  In such situations, the input instructions will state the requirements. Remember that specifying the unit flags does not enforce consistent use of units.  The user must insure that consistent units are used in all input data.

\subsection{Steady-State Simulations}
A steady-state simulation is represented by a single stress period having a single time step with the storage term set to zero. Setting the number and length of stress periods and time steps is the responsibility of the Timing Module of the \mf framework. The length of the stress period and time step will not affect the head solution because the time derivative is not calculated in a steady-state problem. Setting the storage term to zero is the responsibility of the Storage Package. Most other packages need not "know" that a simulation is steady state.

A GWF Model also can be mixed transient and steady state because each stress period can be designated transient or steady state.  Thus, a GWF Model can start with a steady-state stress period and continue with one or more transient stress periods.  The settings for controlling steady-state and transient options are in the Storage Package.  If the Storage Package is not specified for a GWF Model, then the storage terms are zero and the GWF Model will be steady state.

\subsection{Volumetric Budget}
A summary of all inflows (sources) and outflows (sinks) of water is called a water budget.  The water budget for the GWF Model is termed a volumetric budget because volumes of water and volumetric flow rates are involved; thus strictly speaking, a volumetric budget is not a mass balance, although this term has been used in other model reports.  \mf calculates a water budget for the overall model as a check on the acceptability of the solution, and to provide a summary of the sources and sinks of water to the flow system.  The water budget is printed to the GWF Model Listing File for selected time steps.

Numerical solution techniques for simultaneous equations do not always result in a correct answer; in particular, iterative solvers may stop iterating before a sufficiently close approximation to the solution is attained.  A water budget provides an indication of the overall acceptability of the solution.  The system of equations solved by the model actually consists of a flow continuity statement for each model cell.  Continuity should also exist for the total flows into and out of the model---that is, the difference between total inflow and total outflow should equal the total change in storage.  In the model program, the water budget is calculated independently of the equation solution process, and in this sense may provide independent evidence of a valid solution.

The total budget as printed in the output does not include internal flows between model cells---only flows into or out of the model as a whole. For example, flow to or from rivers, flow to or from constant-head cells, and flow to or from wells are all included in the overall budget terms.  Flow into and out of storage is also considered part of the overall budget inasmuch as accumulation in storage effectively removes water from the flow system and storage release effectively adds water to the flow---even though neither process, in itself, involves the transfer of water into or out of the ground-water regime.  Each hydrologic package calculates its own contribution to the budget.

For every time step, the budget subroutine of each hydrologic package calculates the rate of flow into and out of the system due to the process simulated by the package.  The inflows and outflows for each component of flow are stored separately.  Most packages deal with only one such component of flow.  In addition to flow, the volumes of water entering and leaving the model during the time step are calculated as the product of flow rate and time-step length.  Cumulative volumes, from the beginning of the simulation, are then calculated and stored.

The GWF Model uses the inflows, outflows, and cumulative volumes to write the budget to the Listing File at the times requested by the model user.  When a budget is written, the flow rates for the last time step and cumulative volumes from the beginning of simulation are written for each component of flow.  Inflows are written separately from outflows.  Following the convention indicated above, water entering storage is treated as an outflow (that is, as a loss of water from the flow system) while water released from storage is treated as an inflow (that is, a source of water to the flow system).  In addition, total inflow and total outflow are written, as well as the difference between total inflow and outflow.  The difference is then written as a percentage error, calculated using the formula:

\begin{equation}
D = \frac{100 (IN-OUT)}{(IN + OUT) / 2}
\end{equation}

\noindent where $D$ is the percentage error term, $IN$ is the total inflow to the system, and $OUT$ is the total outflow.

If the model equations are solved correctly, the percentage error should be small.  In general, flow rates may be taken as an indication of solution validity for the time step to which they apply, while cumulative volumes are an indication of validity for the entire simulation up to the time of the output.  The budget is written to the GWF Model Listing File at the end of each stress period whether requested or not.

\subsection{Cell-By-Cell Flows}
In some situations, calculating flow terms for various subregions of the model is useful.  To facilitate such calculations, provision has been made to save flow terms for individual cells in a separate binary file so they can be used in computations external to the model itself.  These individual cell flows are referred to here as ``cell-by-cell'' flow terms and are of four general types: (1) cell-by-cell stress flows, or flows into or from an individual cell caused by one of the external stresses represented in the model, such as evapotranspiration or recharge; (2) cell-by-cell storage terms, which give the rate of accumulation or depletion of storage in an individual cell; and (3) internal cell-by-cell flows, which are actually the flows across individual cell faces---that is, between adjacent model cells.  These four kinds of cell-by-cell flow terms are discussed further in subsequent paragraphs.  To save any of these cell-by-cell terms, two flags in the model input must be set.  The input to the Output Control file indicates the time steps for which cell-by-cell terms are to be saved. In addition, each hydrologic package includes an option called SAVE\_FLOWS that must be set if the cell-by-cell terms computed by that package are to be saved.  Thus, if the appropriate option in the Evapotranspiration Package input is set, cell-by-cell evapotranspiration terms will be saved for each time step for which the saving of cell-by-cell flow is requested through the Output Control Option.  Only flow values are saved in the cell-by-cell files; neither water volumes nor cumulative water volumes are included.  The flow dimensions are volume per unit time, where volume and time are in the same units used for all model input data.  The cell-by-cell flow values are stored in unformatted form to make the most efficient use of disk space; see the Budget File section toward the end of this user guide for information on how the data are written to a file.

The cell-by-cell storage term gives the net flow to or from storage in a variable-head cell.  The net storage for each cell in the grid is saved in transient simulations if the appropriate flags are set.  Withdrawal from storage in the cell is considered positive, whereas accumulation in storage is considered negative.

The cell-by-cell constant-head flow term gives the flow into or out of an individual constant-head cell (specified with the CHD Package).  This term is always associated with the constant-head cell itself, rather than with the surrounding cells that contribute or receive the flow.  A constant-head cell may be surrounded by as many as six adjacent variable-head cells for a regular grid or any number of cells for the other grid types.  The cell-by-cell calculation provides a single flow value for each constant-head cell, representing the algebraic sum of the flows between that cell and all of the adjacent variable-head cells.  A positive value indicates that the net flow is away from the constant-head cell (into the variable-head part of the grid); a negative value indicates that the net flow is into the constant-head cell.

The internal cell-by-cell flow values represent flows across the individual faces of a model cell.  Flows between cells are written in the compressed row storage format, whereby the flow between cell $n$ and each one of its connecting $m$ neighbor cells are contained in a single one-dimensional array.  Flows are positive for the cell in question.  Thus the flow reported for cell $n$ and its connection with cell $m$ is opposite in sign to the flow reported for cell $m$ and its connection with cell $n$.  These internal cell-by-cell flow values are useful in calculations of the groundwater flow into various subregions of the model, or in constructing flow vectors.

Cell-by-cell stress flows are flow rates into or out of the model, at a particular cell, owing to one particular external stress.  For example, the cell-by-cell evapotranspiration term for cell $n$ would give the flow out of the model by evapotranspiration from cell $n$.  Cell-by-cell stress flows are considered positive if flow is into the cell, and negative if out of the cell.

\newpage
\subsection{GWF Model Name File}
The GWT Model Name File specifies the options and packages that are active for a GWT model.  The Name File contains two blocks: OPTIONS  and PACKAGES. The length of each line must be 299 characters or less. The lines in each block can be in any order.  Files listed in the PACKAGES block must exist when the program starts. 

Comment lines are indicated when the first character in a line is one of the valid comment characters.  Commented lines can be located anywhere in the file. Any text characters can follow the comment character. Comment lines have no effect on the simulation; their purpose is to allow users to provide documentation about a particular simulation. 

\vspace{5mm}
\subsubsection{Structure of Blocks}
\lstinputlisting[style=blockdefinition]{./mf6ivar/tex/gwt-nam-options.dat}
\lstinputlisting[style=blockdefinition]{./mf6ivar/tex/gwt-nam-packages.dat}

\vspace{5mm}
\subsubsection{Explanation of Variables}
\begin{description}
\input{./mf6ivar/tex/gwt-nam-desc.tex}
\end{description}

\begin{table}[H]
\caption{Ftype values described in this report.  The \texttt{Pname} column indicates whether or not a package name can be provided in the name file}
\small
\begin{center}
\begin{tabular*}{\columnwidth}{l l l}
\hline
\hline
Ftype & Input File Description & \texttt{Pname}\\
\hline
DIS6 & Rectilinear Discretization Input File \\
DISV6 & Discretization by Vertices Input File \\
DISU6 & Unstructured Discretization Input File \\
IC6 & Initial Conditions Package \\
OC6 & Output Control Option \\
ADV6 & Advection Package \\ 
DSP6 & Dispersion Package \\ 
STO6 & Storage Package \\
RCT6 & Reactions Package \\
IMD6 & Immobile Domain Package \\
SSM6 & Source and Sink Mixing Package \\ 
CNC6 & Constant Concentration Package & *\\ 
SRC6 & Mass Source Loading Package & * \\ 
OBS6 & Observations Option \\
\hline 
\end{tabular*}
\label{table:ftype}
\end{center}
\normalsize
\end{table}

\vspace{5mm}
\subsubsection{Example Input File}
\lstinputlisting[style=inputfile]{./mf6ivar/examples/gwt-nam-example.dat}



\newpage
\subsection{Structured Discretization (DIS) Input File}
\input{gwf/dis}

\newpage
\subsection{Discretization with Vertices (DISV) Input File}
\input{gwf/disv}

\newpage
\subsection{Unstructured Discretization (DISU) Input File}
Discretization information for unstructured grids is read from the file that is specified by ``DISU6'' as the file type.  Only one discretization input file (DISU6, DISV6 or DIS6) can be specified for a model.

The shape and position of each cell can be defined using vertices.  This information is optional and is only read if the number of vertices (NVERT) in the DIMENSIONS block is specified and is assigned a value larger than zero.  If the vertices and two-dimensional cell information is provided in this file, then this information is also written to the binary grid file.  Providing this information may be useful for other postprocessing programs that read the binary grid file.

The DISU Package does not support the concept of layers, which is different from the DISU implementation in MODFLOW-USG.  In \mf~all grid input and output for models that use the DISU Package is entered or written as a one-dimensional array of size nodes.

\vspace{5mm}
\subsubsection{Structure of Blocks}
\lstinputlisting[style=blockdefinition]{./mf6ivar/tex/gwf-disu-options.dat}
\lstinputlisting[style=blockdefinition]{./mf6ivar/tex/gwf-disu-dimensions.dat}
\lstinputlisting[style=blockdefinition]{./mf6ivar/tex/gwf-disu-griddata.dat}
\lstinputlisting[style=blockdefinition]{./mf6ivar/tex/gwf-disu-connectiondata.dat}
\lstinputlisting[style=blockdefinition]{./mf6ivar/tex/gwf-disu-vertices.dat}
\lstinputlisting[style=blockdefinition]{./mf6ivar/tex/gwf-disu-cell2d.dat}

\vspace{5mm}
\subsubsection{Explanation of Variables}
\begin{description}
% DO NOT MODIFY THIS FILE DIRECTLY.  IT IS CREATED BY mf6ivar.py 

\item \textbf{Block: OPTIONS}

\begin{description}
\item \texttt{length\_units}---is the length units used for this model.  Values can be ``FEET'', ``METERS'', or ``CENTIMETERS''.  If not specified, the default is ``UNKNOWN''.

\item \texttt{NOGRB}---keyword to deactivate writing of the binary grid file.

\item \texttt{xorigin}---x-position of the origin used for model grid vertices.  This value should be provided in a real-world coordinate system.  A default value of zero is assigned if not specified.  The value for XORIGIN does not affect the model simulation, but it is written to the binary grid file so that postprocessors can locate the grid in space.

\item \texttt{yorigin}---y-position of the origin used for model grid vertices.  This value should be provided in a real-world coordinate system.  If not specified, then a default value equal to zero is used.  The value for YORIGIN does not affect the model simulation, but it is written to the binary grid file so that postprocessors can locate the grid in space.

\item \texttt{angrot}---counter-clockwise rotation angle (in degrees) of the model grid coordinate system relative to a real-world coordinate system.  If not specified, then a default value of 0.0 is assigned.  The value for ANGROT does not affect the model simulation, but it is written to the binary grid file so that postprocessors can locate the grid in space.

\end{description}
\item \textbf{Block: DIMENSIONS}

\begin{description}
\item \texttt{nodes}---is the number of cells in the model grid.

\item \texttt{nja}---is the sum of the number of connections and NODES.  When calculating the total number of connections, the connection between cell n and cell m is considered to be different from the connection between cell m and cell n.  Thus, NJA is equal to the total number of connections, including n to m and m to n, and the total number of cells.

\item \texttt{nvert}---is the total number of (x, y) vertex pairs used to define the plan-view shape of each cell in the model grid.  If NVERT is not specified or is specified as zero, then the VERTICES and CELL2D blocks below are not read.  NVERT and the accompanying VERTICES and CELL2D blocks should be specified for most simulations.  If the XT3D or SAVE\_SPECIFIC\_DISCHARGE options are specified in the NPF Package, these this information is required.

\end{description}
\item \textbf{Block: GRIDDATA}

\begin{description}
\item \texttt{top}---is the top elevation for each cell in the model grid.

\item \texttt{bot}---is the bottom elevation for each cell.

\item \texttt{area}---is the cell surface area (in plan view).

\end{description}
\item \textbf{Block: CONNECTIONDATA}

\begin{description}
\item \texttt{iac}---is the number of connections (plus 1) for each cell.  The sum of all the entries in IAC must be equal to NJA.

\item \texttt{ja}---is a list of cell number (n) followed by its connecting cell numbers (m) for each of the m cells connected to cell n. The number of values to provide for cell n is IAC(n).  This list is sequentially provided for the first to the last cell. The first value in the list must be cell n itself, and the remaining cells must be listed in an increasing order (sorted from lowest number to highest).  Note that the cell and its connections are only supplied for the GWF cells and their connections to the other GWF cells.  Also note that the JA list input may be divided such that every node and its connectivity list can be on a separate line for ease in readability of the file. To further ease readability of the file, the node number of the cell whose connectivity is subsequently listed, may be expressed as a negative number, the sign of which is subsequently converted to positive by the code.

\item \texttt{ihc}---is an index array indicating the direction between node n and all of its m connections.  If IHC = 0 then cell n and cell m are connected in the vertical direction.  Cell n overlies cell m if the cell number for n is less than m; cell m overlies cell n if the cell number for m is less than n.  If IHC = 1 then cell n and cell m are connected in the horizontal direction.  If IHC = 2 then cell n and cell m are connected in the horizontal direction, and the connection is vertically staggered.  A vertically staggered connection is one in which a cell is horizontally connected to more than one cell in a horizontal connection.

\item \texttt{cl12}---is the array containing connection lengths between the center of cell n and the shared face with each adjacent m cell.

\item \texttt{hwva}---is a symmetric array of size NJA.  For horizontal connections, entries in HWVA are the horizontal width perpendicular to flow.  For vertical connections, entries in HWVA are the vertical area for flow.  Thus, values in the HWVA array contain dimensions of both length and area.  Entries in the HWVA array have a one-to-one correspondence with the connections specified in the JA array.  Likewise, there is a one-to-one correspondence between entries in the HWVA array and entries in the IHC array, which specifies the connection type (horizontal or vertical).  Entries in the HWVA array must be symmetric; the program will terminate with an error if the value for HWVA for an n to m connection does not equal the value for HWVA for the corresponding n to m connection.

\item \texttt{angldegx}---is the angle (in degrees) between the horizontal x-axis and the outward normal to the face between a cell and its connecting cells. The angle varies between zero and 360.0 degrees, where zero degrees points in the positive x-axis direction, and 90 degrees points in the positive y-axis direction.  ANGLDEGX is only needed if horizontal anisotropy is specified in the NPF Package, if the XT3D option is used in the NPF Package, or if the SAVE\_SPECIFIC\_DISCHARGE option is specifed in the NPF Package.  ANGLDEGX does not need to be specified if these conditions are not met.  ANGLDEGX is of size NJA; values specified for vertical connections and for the diagonal position are not used.  Note that ANGLDEGX is read in degrees, which is different from MODFLOW-USG, which reads a similar variable (ANGLEX) in radians.

\end{description}
\item \textbf{Block: VERTICES}

\begin{description}
\item \texttt{iv}---is the vertex number.  Records in the VERTICES block must be listed in consecutive order from 1 to NVERT.

\item \texttt{xv}---is the x-coordinate for the vertex.

\item \texttt{yv}---is the y-coordinate for the vertex.

\end{description}
\item \textbf{Block: CELL2D}

\begin{description}
\item \texttt{icell2d}---is the cell2d number.  Records in the CELL2D block must be listed in consecutive order from 1 to NODES.

\item \texttt{xc}---is the x-coordinate for the cell center.

\item \texttt{yc}---is the y-coordinate for the cell center.

\item \texttt{ncvert}---is the number of vertices required to define the cell.  There may be a different number of vertices for each cell.

\item \texttt{icvert}---is an array of integer values containing vertex numbers (in the VERTICES block) used to define the cell.  Vertices must be listed in clockwise order.

\end{description}


\end{description}

\vspace{5mm}
\subsubsection{Example Input File}
\lstinputlisting[style=inputfile]{./mf6ivar/examples/gwf-disu-example.dat}



\newpage
\subsection{Initial Conditions (IC) Package}
Initial Conditions (IC) Package information is read from the file that is specified by ``IC6'' as the file type.  Only one IC Package can be specified for a GWT model. 

\vspace{5mm}
\subsubsection{Structure of Blocks}
%\lstinputlisting[style=blockdefinition]{./mf6ivar/tex/gwf-ic-options.dat}
\lstinputlisting[style=blockdefinition]{./mf6ivar/tex/gwt-ic-griddata.dat}

\vspace{5mm}
\subsubsection{Explanation of Variables}
\begin{description}
% DO NOT MODIFY THIS FILE DIRECTLY.  IT IS CREATED BY mf6ivar.py 

\item \textbf{Block: GRIDDATA}

\begin{description}
\item \texttt{strt}---is the initial (starting) concentration---that is, concentration at the beginning of the GWT Model simulation.  STRT must be specified for all simulations, including steady-state simulations. One value is read for every model cell. For simulations in which the first stress period is steady state, the values used for STRT generally do not affect the simulation. The execution time, however, will be less if STRT includes concentrations that are close to the steady-state solution.

\end{description}


\end{description}

\vspace{5mm}
\subsubsection{Example Input File}
\lstinputlisting[style=inputfile]{./mf6ivar/examples/gwt-ic-example.dat}



\newpage
\subsection{Output Control (OC) Option}
Input to the Output Control Option of the Groundwater Transport Model is read from the file that is specified as type ``OC6'' in the Name File. If no ``OC6'' file is specified, default output control is used. The Output Control Option determines how and when concentrations are printed to the listing file and/or written to a separate binary output file.  Under the default, head and overall budget are written to the Listing File at the end of every stress period. The default printout format for concentrations is 10G11.4.

Output Control data must be specified using words.  The numeric codes supported in earlier MODFLOW versions can no longer be used.

For the PRINT and SAVE options of heads, there is no longer an option to specify individual layers.  Whenever one of these arrays is printed or saved, all layers are printed or saved.

\vspace{5mm}
\subsubsection{Structure of Blocks}
\vspace{5mm}

\noindent \textit{FOR EACH SIMULATION}
\lstinputlisting[style=blockdefinition]{./mf6ivar/tex/gwt-oc-options.dat}
\vspace{5mm}
\noindent \textit{FOR ANY STRESS PERIOD}
\lstinputlisting[style=blockdefinition]{./mf6ivar/tex/gwt-oc-period.dat}

\vspace{5mm}
\subsubsection{Explanation of Variables}
\begin{description}
\input{./mf6ivar/tex/gwt-oc-desc.tex}
\end{description}

\vspace{5mm}
\subsubsection{Example Input File}
\lstinputlisting[style=inputfile]{./mf6ivar/examples/gwt-oc-example.dat}


\newpage
\subsection{Observation (OBS) Utility for a GWF Model}
\input{gwf/gwf-obs}

\newpage
\subsection{Node Property Flow (NPF) Package}
\input{gwf/npf}

\newpage
\subsection{Horizontal Flow Barrier (HFB) Package}
\input{gwf/hfb}

\newpage
\subsection{Storage (STO) Package}
Storage (STO) Package information is read from the file that is specified by ``STO6'' as the file type.  Only one STO Package can be specified for a GWT model. 

\vspace{5mm}
\subsubsection{Structure of Blocks}
\lstinputlisting[style=blockdefinition]{./mf6ivar/tex/gwt-sto-options.dat}
\lstinputlisting[style=blockdefinition]{./mf6ivar/tex/gwt-sto-griddata.dat}

\vspace{5mm}
\subsubsection{Explanation of Variables}
\begin{description}
% DO NOT MODIFY THIS FILE DIRECTLY.  IT IS CREATED BY mf6ivar.py 

\item \textbf{Block: OPTIONS}

\begin{description}
\item \texttt{SAVE\_FLOWS}---keyword to indicate that STO flow terms will be written to the file specified with ``BUDGET FILEOUT'' in Output Control.

\end{description}
\item \textbf{Block: GRIDDATA}

\begin{description}
\item \texttt{porosity}---is the aquifer porosity.

\end{description}


\end{description}

\vspace{5mm}
\subsubsection{Example Input File}
\lstinputlisting[style=inputfile]{./mf6ivar/examples/gwt-sto-example.dat}



\newpage
\subsection{Skeletal Storage, Compaction, and Subsidance (CSUB) Package}
Input to the Skeletal Storage, Compaction, and Subsidance (CSUB) Package is read from the file that has type ``CSUB6'' in the Name File.  If the CSUB Package is not included for a model, then storage changes resulting from compaction will not be calculated.  Only one CSUB Package can be specified for a GWF model.

\vspace{5mm}
\subsubsection{Structure of Blocks}

\vspace{5mm}
\noindent \textit{FOR EACH SIMULATION}
\lstinputlisting[style=blockdefinition]{./mf6ivar/tex/gwf-csub-options.dat}
\lstinputlisting[style=blockdefinition]{./mf6ivar/tex/gwf-csub-dimensions.dat}
\lstinputlisting[style=blockdefinition]{./mf6ivar/tex/gwf-csub-griddata.dat}
\lstinputlisting[style=blockdefinition]{./mf6ivar/tex/gwf-csub-packagedata.dat}
\vspace{5mm}
\noindent \textit{FOR ANY STRESS PERIOD}
\lstinputlisting[style=blockdefinition]{./mf6ivar/tex/gwf-csub-period.dat}
\packageperioddescription

\vspace{5mm}
\subsubsection{Explanation of Variables}
\begin{description}
% DO NOT MODIFY THIS FILE DIRECTLY.  IT IS CREATED BY mf6ivar.py 

\item \textbf{Block: OPTIONS}

\begin{description}
\item \texttt{STRAIN\_CSV\_INTERBED}---keyword to specify the record that corresponds to final interbed strain output.

\item \texttt{FILEOUT}---keyword to specify that an output filename is expected next.

\item \texttt{interbedstrainfile}---name of the comma-separated-values output file to write final interbed strain information.

\item \texttt{STRAIN\_CSV\_SKELETAL}---keyword to specify the record that corresponds to final skeletal strain output.

\item \texttt{skeletalstrainfile}---name of the comma-separated-values output file to write final skeletal strain information.

\item \texttt{SAVE\_FLOWS}---keyword to indicate that cell-by-cell flow terms will be written to the file specified with ``BUDGET SAVE FILE'' in Output Control.

\item \texttt{gammaw}---specific gravity of water. For freshwater, GAMMAW is 9806.65 Newtons/cubic meters or 62.48 lb/cubic foot in SI and English units, respectively. By default, GAMMAW is 9806.65.

\item \texttt{beta}---compressibility of water. Typical values of BETA are 4.6512e-10 1/Pa or 2.2270e-8 lb/square foot in SI and English units, respectively. By default, BETA is 4.6512e-10.

\item \texttt{time\_weight}---effective stress time weight used to calculate the storage parameters. A TIME\_WEIGHT value of 1 results in the use of the current estimate of the effective stress to calculate storage parameters. A TIME\_WEIGHT value of 0 results in the use of the effective stress from the previous time step to calculate storage parameters. Any TIME\_FACTOR value can be specified if HEAD\_BASED is specified in the OPTIONS block.  TIME\_WEIGHT must be either 0 or 1 if HEAD\_BASED is not specified in the OPTIONS block. By default, TIME\_FACTOR is 1.

\item \texttt{HEAD\_BASED}---keyword to indicate the head-based formulation will be used to simulate coarse-grained aquifer materials and no-delay and delay interbeds.

\item \texttt{DELAY\_FULL\_CELL}---keyword to indicate that head-based delay interbeds will be simulated using a full-cell approach.

\item \texttt{ndelaycells}---number of nodes used to discretize the delay interbed. If not specified, then a default value of 19 is assigned.

\item \texttt{DELAY\_SATURATION\_SCALING}---keyword to indicate that delay interbeds can be assigned to convertible cells. By default, delay interbeds are not allowed in convertible cells. The program will terminate with an error if DELAY\_SATURATION\_SCALING is not specified and delay intebeds are assigned to convertible cells. The program will also terminate if DELAY\_SATURATION\_SCALING is specified and water-levels fall below the top of the interbed during a simulation.

\item \texttt{SPECIFIED\_INITIAL\_INTERBED\_STATE}---keyword to indicate that absolute preconsolidation stresses (heads) and delay bed heads will be specified for interbeds defined in the PACKAGEDATA block. The SPECIFIED\_INITIAL\_INTERBED\_STATE option is equivalent to specifying the SPECIFIED\_INITIAL\_PRECONSOLITATION\_STRESS and SPECIFIED\_INITIAL\_DELAY\_HEAD. If SPECIFIED\_INITIAL\_INTERBED\_STATE is not specified then preconsolidation stress (head) and delay bed head values specified in the PACKAGEDATA block are relative to simulated values if the first stress period is steady-state or initial stresses and GWF heads if the first stress period is transient.

\item \texttt{SPECIFIED\_INITIAL\_PRECONSOLIDATION\_STRESS}---keyword to indicate that absolute preconsolidation stresses (heads) will be specified for interbeds defined in the PACKAGEDATA block. If SPECIFIED\_INITIAL\_PRECONSOLITATION\_STRESS and SPECIFIED\_INITIAL\_INTERBED\_STATE are not specified then preconsolidation stress (head) values specified in the PACKAGEDATA block are relative to simulated values if the first stress period is steady-state or initial stresses (heads) if the first stress period is transient.

\item \texttt{SPECIFIED\_INITIAL\_DELAY\_HEAD}---keyword to indicate that absolute initial delay bed head will be specified for interbeds defined in the PACKAGEDATA block. If SPECIFIED\_INITIAL\_DELAY\_HEAD and SPECIFIED\_INITIAL\_INTERBED\_STATE are not specified then delay bead head values specified in the PACKAGEDATA block are relative to simulated values if the first stress period is steady-state or initial GWF heads if the first stress period is transient.

\item \texttt{GEO\_STRESS\_OFFSET}---keyword to indicate that a geostatic stress offset will be specified in the PERIOD block.

\item \texttt{COMPRESSION\_INDICES}---keyword to indicate that the the recompression (CR) and compresion (CC) indices are specified instead of the elastic skeletal specific storage (SSE) and inelastic skeletal specific storage (SSV) coefficients.

\item \texttt{UPDATE\_MATERIAL\_PROPERTIES}---keyword to indicate that the thickness and void ratio of skeletal and no delay interbed sediments will not vary during the simulation.

\item \texttt{CELL\_FRACTION}---keyword to indicate that the thickness of interbeds will be specified in terms of the fraction of cell thickness.

\item \texttt{COMPACTION}---keyword to specify that record corresponds to the compaction.

\item \texttt{compactionfile}---name of the binary output file to write compaction information.

\item \texttt{TS6}---keyword to specify that record corresponds to a time-series file.

\item \texttt{FILEIN}---keyword to specify that an input filename is expected next.

\item \texttt{ts6\_filename}---defines a time-series file defining time series that can be used to assign time-varying values. See the ``Time-Variable Input'' section for instructions on using the time-series capability.

\item \texttt{OBS6}---keyword to specify that record corresponds to an observations file.

\item \texttt{obs6\_filename}---name of input file to define observations for the CSUB package. See the ``Observation utility'' section for instructions for preparing observation input files. Table \ref{table:obstype} lists observation type(s) supported by the CSUB package.

\end{description}
\item \textbf{Block: DIMENSIONS}

\begin{description}
\item \texttt{ninterbeds}---is the number of CSUB interbed systems.  More than 1 CSUB interbed systems can be assigned to a GWF cell; however, only 1 GWF cell can be assigned to a single CSUB interbed system.

\end{description}
\item \textbf{Block: GRIDDATA}

\begin{description}
\item \texttt{ske\_cr}---is elastic skeletal specific storage.

\item \texttt{sk\_theta}---is the initial porosity of the aquifer.

\item \texttt{sgm}---is specific gravity of moist or unsaturated sediments.

\item \texttt{sgs}---is specific gravity of saturated sediments.

\end{description}
\item \textbf{Block: PACKAGEDATA}

\begin{description}
\item \texttt{icsubno}---integer value that defines the CSUB interbed number associated with the specified PACKAGEDATA data on the line. CSUBNO must be greater than zero and less than or equal to NCSUBCELLS.  CSUB information must be specified for every CSUB cell or the program will terminate with an error.  The program will also terminate with an error if information for a CSUB intebed number is specified more than once.

\item \texttt{cellid}---is the cell identifier, and depends on the type of grid that is used for the simulation.  For a structured grid that uses the DIS input file, CELLID is the layer, row, and column.   For a grid that uses the DISV input file, CELLID is the layer and CELL2D number.  If the model uses the unstructured discretization (DISU) input file, CELLID is the node number for the cell.

\item \texttt{cdelay}---character string that defines the subsidence delay type for the CSUB cell. Possible subsidence package CDELAY strings include: NODELAY--character keyword to indicate that delay will not be simulated in the CSUB cell.  DELAY--character keyword to indicate that delay will be simulated in the CSUB cell.

\item \texttt{pcs0}---is the initial offset from the calculated initial effective stress or initial preconsolidation stress in the CSUB interbed, in units of height of a column of water. PCS0 is the initial preconsolidation stress if SPECIFIED\_INITIAL\_INTERBED\_STATE or SPECIFIED\_INITIAL\_PRECONSOLIDATION\_STRESS are specified in the OPTIONS block.

\item \texttt{thick\_frac}---is the interbed thickness or cell fraction of the CSUB interbed. Interbed thickness is specified as a fraction of the cell thickness if CELL\_FRACTION is specified in the OPTIONS block.

\item \texttt{rnb}---is the interbed material factor equivalent number of interbeds for the system of delay interbeds for interbed CSUBNO. RNB must be greater than or equal to 1 if CDELAY is DELAY. Otherwise, RNB can be any value.

\item \texttt{ssv\_cc}---is the initial inelastic skeletal specific storage or compression index in the CSUB interbed. The initial inelastic skeletal specific storage is specified if STORAGE\_COEFFICIENT is specified in the OPTIONS block.

\item \texttt{sse\_cr}---is the initial elastic skeletal specific storage or recompression index in the CSUB interbed. The initial elastic skeletal specific storage is specified if STORAGE\_COEFFICIENT is specified in the OPTIONS block.

\item \texttt{theta}---is the initial porosity of the CSUB interbed.

\item \texttt{kv}---is the vertical hydraulic conductivity of the CSUB interbed. KV must be greater than 0 if CDELAY is DELAY. Otherwise, KV can be any value.

\item \texttt{h0}---is the initial offset from the head in cell cellid or the initial head in the CSUB interbed. H0 can be any value if CDELAY is NODELAY. H0 is the initial head in the delay bed if SPECIFIED\_INITIAL\_INTERBED\_STATE or SPECIFIED\_INITIAL\_DELAY\_HEAD are specified in the OPTIONS block.

\item \texttt{boundname}---name of the CSUB cell cell.  BOUNDNAME is an ASCII character variable that can contain as many as 40 characters.  If BOUNDNAME contains spaces in it, then the entire name must be enclosed within single quotes.

\end{description}
\item \textbf{Block: PERIOD}

\begin{description}
\item \texttt{iper}---integer value specifying the starting stress period number for which the data specified in the PERIOD block apply.  IPER must be less than or equal to NPER in the TDIS Package and greater than zero.  The IPER value assigned to a stress period block must be greater than the IPER value assigned for the previous PERIOD block.  The information specified in the PERIOD block will continue to apply for all subsequent stress periods, unless the program encounters another PERIOD block.

\item \texttt{sig0}---is the stress offset for the cell. SIG0 is added to the calculated geostatic stress for the cell. SIG0 is specified only if GEO\_STRESS\_OFFSET is specified in the OPTIONS block.

\end{description}


\end{description}

\vspace{5mm}
\subsubsection{Example Input File}
\lstinputlisting[style=inputfile]{./mf6ivar/examples/gwf-csub-example.dat}


\vspace{5mm}
\subsubsection{Available observation types}
Subsidence Package observations include all of the terms that contribute to the continuity equation for each GWF cell. The data required for each CSUB Package observation type is defined in table~\ref{table:gwf-csubobstype}. Negative and positive values for \texttt{CSUB} observations represent a loss from and gain to the GWF model, respectively.


\begin{longtable}{p{2cm} p{2.75cm} p{2cm} p{1.25cm} p{7cm}}
\caption{Available CSUB Package observation types} \tabularnewline

\hline
\hline
\textbf{Stress Package} & \textbf{Observation type} & \textbf{ID} & \textbf{ID2} & \textbf{Description} \\
\hline
\endfirsthead

\captionsetup{textformat=simple}
\caption*{\textbf{Table \arabic{table}.}{\quad}Available CSUB Package observation types.---Continued} \\

\hline
\hline
\textbf{Stress Package} & \textbf{Observation type} & \textbf{ID} & \textbf{ID2} & \textbf{Description} \\
\hline
\endhead

\hline
\endfoot

CSUB & csub & icsubno or boundname & -- & Flow between the groundwater system and a interbed or group of interbeds. \\
CSUB & inelastic-csub & icsubno or boundname & -- & Flow between the groundwater system and a interbed or group of interbeds from inelastic compaction. \\
CSUB & elastic-csub & icsubno or boundname & -- & Flow between the groundwater system and a interbed or group of interbeds from elastic compaction. \\
CSUB & skeletal-csub & cellid & -- & Flow between the groundwater system and coarse skeletal materials in a GWF cell. \\
CSUB & csub-cell & cellid & -- & Flow between the groundwater system for all interbeds and coarse skeletal materials in a GWF cell. \\
CSUB & wcomp-csub-cell & cellid & -- & Flow between the groundwater system for all interbeds and coarse skeletal materials in a GWF cell from water compressibility. \\

CSUB & sk & icsubno or boundname & -- & Convertible interbed storativity in a interbed or group of interbeds. Convertible interbed storativity is inelastic interbed storativity if the current effective stress is greater than the preconsolidation stress. \\
CSUB & ske & icsubno or boundname & -- & Elastic interbed storativity in a interbed or group of interbeds. \\
CSUB & sk-cell & cellid & -- & Convertible interbed and skeletal storativity in a GWF cell. Convertible interbed storativity is inelastic interbed storativity if the current effective stress is greater than the preconsolidation stress. \\
CSUB & ske-cell & cellid & -- & Elastic interbed and skeletal storativity in a GWF cell. \\

CSUB & estress-cell & cellid & -- & effective stress in a GWF cell. \\
CSUB & gstress-cell & cellid & -- & geostatic stress in a GWF cell. \\

CSUB & interbed-compaction & icsubno or boundname  & -- & interbed compaction in a interbed or group of interbeds. \\
CSUB & inelastic-compaction &  icsubno or boundname & -- & inelastic interbed compaction in a interbed or group of interbeds. \\
CSUB & elastic-compaction &  icsubno or boundname & -- & elastic interbed compaction a interbed or group of interbeds. \\
CSUB & skeletal-compaction & cellid  & -- & elastic compaction in coarse skeletal materials in a GWF cell. \\
CSUB & compaction-cell & cellid  & -- & total compaction in coarse skeletal materials and all interbeds in a GWF cell. \\

CSUB & thickness & icsubno or boundname & -- & thickness of a interbed or group of interbeds. \\
CSUB & skeletal-thickness & cellid  & -- & thickness of coarse skeletal materials in a GWF cell. \\
CSUB & thickness-cell & cellid  & -- & total thickness of coarse skeletal materials and all interbeds in a GWF cell. \\

CSUB & theta & icsubno & -- & porosity of a interbed . \\
CSUB & skeletal-theta & cellid  & -- & porosity of coarse skeletal materials in a GWF cell. \\
CSUB & theta-cell & cellid  & -- & thickness-weighted porosity of coarse skeletal materials and all interbeds in a GWF cell. \\

CSUB & delay-head & icsubno  & idcellno & head in interbed delay cell idcellno (1 $<=$ idcellno $<=$ NDELAYCELLS). \\
CSUB & preconstress & icsubno  & idcellno & preconsolidation stress in interbed delay cell idcellno (1 $<=$ idcellno $<=$ NDELAYCELLS). \\
CSUB & preconstress-cell & cellid  & -- & preconsolidation stress in a GWF cell containing at least one interbed. \\

CSUB & delay-flowtop & icsubno  & -- & Flow between the groundwater system and a delay interbed across the top of the interbed. \\
CSUB & delay-flowbot & icsubno  & -- & Flow between the groundwater system and a delay interbed across the bottom of the interbed. \\


\label{table:gwf-csubobstype}
\end{longtable}

\vspace{5mm}
\subsubsection{Example Observation Input File}
\lstinputlisting[style=inputfile]{./mf6ivar/examples/gwf-csub-example-obs.dat}


\newpage
\subsection{Constant-Head (CHD) Package}
\input{gwf/chd}

\newpage
\subsection{Well (WEL) Package}
\input{gwf/wel}

\newpage
\subsection{Drain (DRN) Package}
\input{gwf/drn}

\newpage
\subsection{River (RIV) Package}
\input{gwf/riv}

\newpage
\subsection{General-Head Boundary (GHB) Package}
\input{gwf/ghb}

\newpage
\subsection{Recharge (RCH) Package -- List-Based Input}
\input{gwf/rch}

\newpage
\subsection{Recharge (RCH) Package -- Array-Based Input}
\input{gwf/rcha}

\newpage
\subsection{Evapotranspiration (EVT) Package -- List-Based Input}
Input to the Evapotranspiration (EVT) Package is read from the file that has type ``EVT6'' in the Name File. Any number of EVT Packages can be specified for a single groundwater flow model. All single-valued variables are free format.

Evapotranspiration input can be specified using lists or arrays, unless the DISU Package is used.  List-based input must be used if discretization is specified using the DISU Package.  List-based input for evapotranspiration is the default, and is described here.  Instructions for specifying array-based evapotranspiration are described in the next section. 

List-based input offers several advantages over the array-based input for specifying evapotranspiration.  First, multiple list entries can be specified for a single cell.  This makes it possible to divide a cell into multiple areas, and assign a different evapotranspiration rate or extinction depth for each area (perhaps based on vegetation type or some other criteria).  In this case, the user would likely specify an auxiliary variable to serve as a multiplier.  This multiplier would be calculated by the user and provided in the input file as the fractional cell are for the individual evapotranspiration entries.  Another advantage to using list-based input for specifying evapotranspiration is that boundnames can be specified.  Boundnames work with the Observations capability and can be used to sum evapotranspiration rates for entries with the same boundname.  A disadvantage of the list-based input is that one cannot easily assign evapotranspiration to the entire model without specifying a list of model cells.  For this reason \mf also supports array-based input for evapotranspiration.

ET input is read in list form, as shown in the PERIOD block below. Each line in the PERIOD block defines all input for one cell. Entries following \texttt{cellid}, in order, define the ET surface (\texttt{etss}), maximum ET flux rate (\texttt{etsr}), extinction depth (\texttt{etsx}), all (\texttt{netseg} -- 1) \texttt{pxdp} values, all (\texttt{netseg} -- 1) \texttt{petm} values, all auxiliary variables (if AUXILIARY option is specified), and boundary name (if BOUNDNAMES option is specified).

\vspace{5mm}
\subsubsection{Structure of Blocks}
\vspace{5mm}

\noindent \textit{FOR EACH SIMULATION}
\lstinputlisting[style=blockdefinition]{./mf6ivar/tex/gwf-evt-options.dat}
\lstinputlisting[style=blockdefinition]{./mf6ivar/tex/gwf-evt-dimensions.dat}
\vspace{5mm}
\noindent \textit{FOR ANY STRESS PERIOD}
\lstinputlisting[style=blockdefinition]{./mf6ivar/tex/gwf-evt-period.dat}
\packageperioddescription

\vspace{5mm}
\subsubsection{Explanation of Variables}
\begin{description}
\input{./mf6ivar/tex/gwf-evt-desc.tex}
\end{description}

\vspace{5mm}
\subsubsection{Example Input File}
\lstinputlisting[style=inputfile]{./mf6ivar/examples/gwf-evt-example.dat}

\vspace{5mm}
\subsubsection{Available observation types}
Evapotranspiration Package observations are limited to the simulated evapotranspiration flow rates (\texttt{evt}). The data required for the EVT Package observation type is defined in table~\ref{table:gwf-evtobstype}. Negative and positive values for an observation represent a loss from and gain to the GWF model, respectively.

\begin{longtable}{p{2cm} p{2.75cm} p{2cm} p{1.25cm} p{7cm}}
\caption{Available EVT Package observation types} \tabularnewline

\hline
\hline
\textbf{Stress Package} & \textbf{Observation type} & \textbf{ID} & \textbf{ID2} & \textbf{Description} \\
\hline
\endhead

\hline
\endfoot

\input{../Common/gwf-evtobs.tex}
\label{table:gwf-evtobstype}
\end{longtable}

\vspace{5mm}
\subsubsection{Example Observation Input File}
\lstinputlisting[style=inputfile]{./mf6ivar/examples/gwf-evt-example-obs.dat}


\newpage
\subsection{Evapotranspiration (EVT) Package -- Array-Based Input}
\input{gwf/evta}

\newpage
\subsection{Multi-Aquifer Well (MAW) Package}
\input{gwf/maw}

\newpage
\subsection{Streamflow Routing (SFR) Package}
\input{gwf/sfr}

\newpage
\subsection{Lake (LAK) Package}
\input{gwf/lak}

\newpage
\subsection{Unsaturated Zone Flow (UZF) Package}
\input{gwf/uzf}

\newpage
\subsection{Water Mover (MVR) Package}
\input{gwf/mvr}

\newpage
\subsection{Ghost-Node Correction (GNC) Package}
\input{gwf/gnc}

\newpage
\subsection{Groundwater Flow (GWF) Exchange}
\input{gwf/gwf-gwf}

