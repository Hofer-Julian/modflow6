\mf is run from the command line by entering the name of the \mf executable program.  If the run is successful, it will conclude with a statement about normal termination.

{\small
\begin{lstlisting}[style=modeloutput]
                                   MODFLOW 6
                U.S. GEOLOGICAL SURVEY MODULAR HYDROLOGIC MODEL
                        VERSION mf6.0.2.34 June 24, 2018

   MODFLOW 6 compiled Jun 27 2018 13:15:35 with IFORT compiler (ver. 18.0.3)

This software is preliminary or provisional and is subject to
revision. It is being provided to meet the need for timely best
science. The software has not received final approval by the U.S.
Geological Survey (USGS). No warranty, expressed or implied, is made
by the USGS or the U.S. Government as to the functionality of the
software and related material nor shall the fact of release
constitute any such warranty. The software is provided on the
condition that neither the USGS nor the U.S. Government shall be held
liable for any damages resulting from the authorized or unauthorized
use of the software.

 Run start date and time (yyyy/mm/dd hh:mm:ss): 2018/06/27 13:18:10

 Writing simulation list file: mfsim.lst
 Using Simulation name file: mfsim.nam
 Solving:  Stress period:     1    Time step:     1
 Run end date and time (yyyy/mm/dd hh:mm:ss): 2018/06/27 13:18:10
 Elapsed run time:  0.283 Seconds

 Normal termination of simulation.

\end{lstlisting}
}


\noindent \mf includes a number of switches that can be passed to the program in order to get additional information.  The available switches can be found by running \mf with the -h switch, for help.  In this case \mf will produce the following.

{\small
\begin{lstlisting}[style=modeloutput]
mf6.exe - MODFLOW mf6.0.2.34 June 24, 2018 (compiled Jun 27 2018 13:15:35)
usage: mf6.exe               run MODFLOW 6 using "mfsim.nam"
   or: mf6.exe [options]     retrieve program information

Options   GNU long option   Meaning
 -h, -?   --help            Show this message
 -v       --version         Display program version information.
 -dev     --develop         Display program develop option mode.
 -c       --compiler        Display compiler information.

Bug reporting and contributions are welcome from the community.
Questions can be asked on the issues page[1]. Before creating a new
issue, please take a moment to search and make sure a similar issue
does not already exist. If one does exist, you can comment (most
simply even with just :+1:) to show your support for that issue.

[1] https://github.com/MODFLOW-USGS/modflow6/issues


\end{lstlisting}
}


\noindent \mf requires that a simulation name file (described in a subsequent section titled ``Simulation Name File'') be present in the working directory.  This simulation name file must be named ``mfsim.nam''.  If the mfsim.nam file is not located in the present working directory, then \mf will terminate with the following error.  

{\small
\begin{lstlisting}[style=modeloutput]

ERROR REPORT:

 mf6.exe: mfsim.nam is not present in working directory.
 Stopping due to error(s)
2

\end{lstlisting}
}


During execution \mf creates a simulation output file, called a listing file, with the name ``mfsim.lst''.  This file contains general simulation information, including information about exchanges between models, timing, and solver progress.  Separate listing files are also written for each individual model.  These listing files contains the details for the specific models.

In the event that \mf encounters an error, the error message is written to the command line window as well as to the simulation listing file.  The error message will also contain the name of the file that was being read when the error occurred, if possible.  This information can be used to diagnose potential causes of the error.  
