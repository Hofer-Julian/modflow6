% DO NOT MODIFY THIS FILE DIRECTLY.  IT IS CREATED BY mf6ivar.py 

\item \textbf{Block: OPTIONS}

\begin{description}
\item \texttt{STRAIN\_CSV\_INTERBED}---keyword to specify the record that corresponds to final interbed strain output.

\item \texttt{FILEOUT}---keyword to specify that an output filename is expected next.

\item \texttt{interbedstrain\_filename}---name of the comma-separated-values output file to write final interbed strain information.

\item \texttt{STRAIN\_CSV\_SKELETAL}---keyword to specify the record that corresponds to final skeletal strain output.

\item \texttt{skeletalstrain\_filename}---name of the comma-separated-values output file to write final skeletal strain information.

\item \texttt{SAVE\_FLOWS}---keyword to indicate that cell-by-cell flow terms will be written to the file specified with ``BUDGET SAVE FILE'' in Output Control.

\item \texttt{gammaw}---specific gravity of water. For freshwater, GAMMAW is 9806.65 Newtons/cubic meters or 62.48 lb/cubic foot in SI and English units, respectively. By default, GAMMAW is 9806.65.

\item \texttt{beta}---compressibility of water. Typical values of BETA are 4.6512e-10 1/Pa or 2.2270e-8 lb/square foot in SI and English units, respectively. By default, BETA is 4.6512e-10.

\item \texttt{time\_weight}---effective stress time weight used to calculate the storage parameters. A TIME\_WEIGHT value of 1 results in the use of the current estimate of the effective stress to calculate storage parameters. A TIME\_WEIGHT value of 0 results in the use of the effective stress from the previous time step to calculate storage parameters. Any TIME\_FACTOR value can be specified if HEAD\_BASED is specified in the OPTIONS block.  TIME\_WEIGHT must be either 0 or 1 if HEAD\_BASED is not specified in the OPTIONS block. By default, TIME\_FACTOR is 1.

\item \texttt{HEAD\_BASED}---keyword to indicate the head-based formulation will be used to simulate coarse-grained aquifer materials and no-delay and delay interbeds.

\item \texttt{DELAY\_FULL\_CELL}---keyword to indicate that head-based delay interbeds will be simulated using a full-cell approach.

\item \texttt{ndelaycells}---number of nodes used to discretize the delay interbed. If not specified, then a default value of 19 is assigned.

\item \texttt{DELAY\_SATURATION\_SCALING}---keyword to indicate that delay interbeds can be assigned to convertible cells. By default, delay interbeds are not allowed in convertible cells. The program will terminate with an error if DELAY\_SATURATION\_SCALING is not specified and delay intebeds are assigned to convertible cells. The program will also terminate if DELAY\_SATURATION\_SCALING is specified and water-levels fall below the top of the interbed during a simulation.

\item \texttt{SPECIFIED\_INITIAL\_INTERBED\_STATE}---keyword to indicate that absolute preconsolidation stresses (heads) and delay bed heads will be specified for interbeds defined in the PACKAGEDATA block. The SPECIFIED\_INITIAL\_INTERBED\_STATE option is equivalent to specifying the SPECIFIED\_INITIAL\_PRECONSOLITATION\_STRESS and SPECIFIED\_INITIAL\_DELAY\_HEAD. If SPECIFIED\_INITIAL\_INTERBED\_STATE is not specified then preconsolidation stress (head) and delay bed head values specified in the PACKAGEDATA block are relative to simulated values if the first stress period is steady-state or initial stresses and GWF heads if the first stress period is transient.

\item \texttt{SPECIFIED\_INITIAL\_PRECONSOLIDATION\_STRESS}---keyword to indicate that absolute preconsolidation stresses (heads) will be specified for interbeds defined in the PACKAGEDATA block. If SPECIFIED\_INITIAL\_PRECONSOLITATION\_STRESS and SPECIFIED\_INITIAL\_INTERBED\_STATE are not specified then preconsolidation stress (head) values specified in the PACKAGEDATA block are relative to simulated values if the first stress period is steady-state or initial stresses (heads) if the first stress period is transient.

\item \texttt{SPECIFIED\_INITIAL\_DELAY\_HEAD}---keyword to indicate that absolute initial delay bed head will be specified for interbeds defined in the PACKAGEDATA block. If SPECIFIED\_INITIAL\_DELAY\_HEAD and SPECIFIED\_INITIAL\_INTERBED\_STATE are not specified then delay bead head values specified in the PACKAGEDATA block are relative to simulated values if the first stress period is steady-state or initial GWF heads if the first stress period is transient.

\item \texttt{GEO\_STRESS\_OFFSET}---keyword to indicate that a geostatic stress offset will be specified in the PERIOD block.

\item \texttt{COMPRESSION\_INDICES}---keyword to indicate that the the recompression (CR) and compresion (CC) indices are specified instead of the elastic skeletal specific storage (SSE) and inelastic skeletal specific storage (SSV) coefficients.

\item \texttt{UPDATE\_MATERIAL\_PROPERTIES}---keyword to indicate that the thickness and void ratio of skeletal and no delay interbed sediments will not vary during the simulation.

\item \texttt{CELL\_FRACTION}---keyword to indicate that the thickness of interbeds will be specified in terms of the fraction of cell thickness.

\item \texttt{COMPACTION}---keyword to specify that record corresponds to the compaction.

\item \texttt{compaction\_filename}---name of the binary output file to write compaction information.

\item \texttt{COMPACTION\_ELASTIC}---keyword to specify that record corresponds to the elastic interbed compaction binary file.

\item \texttt{elastic\_compaction\_filename}---name of the binary output file to write elastic interbed compaction information.

\item \texttt{COMPACTION\_INELASTIC}---keyword to specify that record corresponds to the inelastic interbed compaction binary file.

\item \texttt{inelastic\_compaction\_filename}---name of the binary output file to write inelastic interbed compaction information.

\item \texttt{COMPACTION\_SKELETAL}---keyword to specify that record corresponds to the elastic skeletal compaction binary file.

\item \texttt{skeletal\_compaction\_filename}---name of the binary output file to write elastic skeletal compaction information.

\item \texttt{ZDISPLACEMENT}---keyword to specify that record corresponds to the z-displacement binary file.

\item \texttt{zdisplacement\_filename}---name of the binary output file to write z-displacement information.

\item \texttt{TS6}---keyword to specify that record corresponds to a time-series file.

\item \texttt{FILEIN}---keyword to specify that an input filename is expected next.

\item \texttt{ts6\_filename}---defines a time-series file defining time series that can be used to assign time-varying values. See the ``Time-Variable Input'' section for instructions on using the time-series capability.

\item \texttt{OBS6}---keyword to specify that record corresponds to an observations file.

\item \texttt{obs6\_filename}---name of input file to define observations for the CSUB package. See the ``Observation utility'' section for instructions for preparing observation input files. Table \ref{table:obstype} lists observation type(s) supported by the CSUB package.

\end{description}
\item \textbf{Block: DIMENSIONS}

\begin{description}
\item \texttt{ninterbeds}---is the number of CSUB interbed systems.  More than 1 CSUB interbed systems can be assigned to a GWF cell; however, only 1 GWF cell can be assigned to a single CSUB interbed system.

\end{description}
\item \textbf{Block: GRIDDATA}

\begin{description}
\item \texttt{ske\_cr}---is elastic skeletal specific storage.

\item \texttt{sk\_theta}---is the initial porosity of the aquifer.

\item \texttt{sgm}---is specific gravity of moist or unsaturated sediments.

\item \texttt{sgs}---is specific gravity of saturated sediments.

\end{description}
\item \textbf{Block: PACKAGEDATA}

\begin{description}
\item \texttt{icsubno}---integer value that defines the CSUB interbed number associated with the specified PACKAGEDATA data on the line. CSUBNO must be greater than zero and less than or equal to NCSUBCELLS.  CSUB information must be specified for every CSUB cell or the program will terminate with an error.  The program will also terminate with an error if information for a CSUB intebed number is specified more than once.

\item \texttt{cellid}---is the cell identifier, and depends on the type of grid that is used for the simulation.  For a structured grid that uses the DIS input file, CELLID is the layer, row, and column.   For a grid that uses the DISV input file, CELLID is the layer and CELL2D number.  If the model uses the unstructured discretization (DISU) input file, CELLID is the node number for the cell.

\item \texttt{cdelay}---character string that defines the subsidence delay type for the CSUB cell. Possible subsidence package CDELAY strings include: NODELAY--character keyword to indicate that delay will not be simulated in the CSUB cell.  DELAY--character keyword to indicate that delay will be simulated in the CSUB cell.

\item \texttt{pcs0}---is the initial offset from the calculated initial effective stress or initial preconsolidation stress in the CSUB interbed, in units of height of a column of water. PCS0 is the initial preconsolidation stress if SPECIFIED\_INITIAL\_INTERBED\_STATE or SPECIFIED\_INITIAL\_PRECONSOLIDATION\_STRESS are specified in the OPTIONS block.

\item \texttt{thick\_frac}---is the interbed thickness or cell fraction of the CSUB interbed. Interbed thickness is specified as a fraction of the cell thickness if CELL\_FRACTION is specified in the OPTIONS block.

\item \texttt{rnb}---is the interbed material factor equivalent number of interbeds for the system of delay interbeds for interbed CSUBNO. RNB must be greater than or equal to 1 if CDELAY is DELAY. Otherwise, RNB can be any value.

\item \texttt{ssv\_cc}---is the initial inelastic skeletal specific storage or compression index in the CSUB interbed. The initial inelastic skeletal specific storage is specified if STORAGE\_COEFFICIENT is specified in the OPTIONS block.

\item \texttt{sse\_cr}---is the initial elastic skeletal specific storage or recompression index in the CSUB interbed. The initial elastic skeletal specific storage is specified if STORAGE\_COEFFICIENT is specified in the OPTIONS block.

\item \texttt{theta}---is the initial porosity of the CSUB interbed.

\item \texttt{kv}---is the vertical hydraulic conductivity of the CSUB interbed. KV must be greater than 0 if CDELAY is DELAY. Otherwise, KV can be any value.

\item \texttt{h0}---is the initial offset from the head in cell cellid or the initial head in the CSUB interbed. H0 can be any value if CDELAY is NODELAY. H0 is the initial head in the delay bed if SPECIFIED\_INITIAL\_INTERBED\_STATE or SPECIFIED\_INITIAL\_DELAY\_HEAD are specified in the OPTIONS block.

\item \texttt{boundname}---name of the CSUB cell cell.  BOUNDNAME is an ASCII character variable that can contain as many as 40 characters.  If BOUNDNAME contains spaces in it, then the entire name must be enclosed within single quotes.

\end{description}
\item \textbf{Block: PERIOD}

\begin{description}
\item \texttt{iper}---integer value specifying the starting stress period number for which the data specified in the PERIOD block apply.  IPER must be less than or equal to NPER in the TDIS Package and greater than zero.  The IPER value assigned to a stress period block must be greater than the IPER value assigned for the previous PERIOD block.  The information specified in the PERIOD block will continue to apply for all subsequent stress periods, unless the program encounters another PERIOD block.

\item \texttt{sig0}---is the stress offset for the cell. SIG0 is added to the calculated geostatic stress for the cell. SIG0 is specified only if GEO\_STRESS\_OFFSET is specified in the OPTIONS block.

\end{description}

