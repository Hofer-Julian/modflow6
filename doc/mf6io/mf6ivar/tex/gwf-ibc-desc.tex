% DO NOT MODIFY THIS FILE DIRECTLY.  IT IS CREATED BY mf6ivar.py 

\item \textbf{Block: OPTIONS}

\begin{description}
\item \texttt{SAVE\_FLOWS}---keyword to indicate that cell-by-cell flow terms will be written to the file specified with ``BUDGET SAVE FILE'' in Output Control.

\item \texttt{HEAD\_BASED}---keyword to indicate the head-based formulation will be used to simulate no-delay and delay interbeds.

\item \texttt{DELAY\_FULL\_CELL}---keyword to indicate that head-based delay interbeds will be simulated using a full-cell approach.

\item \texttt{ndelaycells}---number of nodes used to discretize the the delay intebed thickness to approximate the head distributions in systems of delay interbeds. If not specified, then a default value of 19 is assigned.

\item \texttt{INTERBED\_STRESS\_OFFSET}---keyword to indicate that an initial stress offset specified in the GRIDDATA block will be applied to each interbed.

\item \texttt{GEO\_STRESS\_OFFSET}---keyword to indicate that a geostatic stress offset will be specified in the PERIOD block.

\item \texttt{COMPRESSION\_INDICES}---keyword to indicate that the the recompression (CR) and compresion (CC) indices are specified instead of the elastic skeletal specific storage (SSE) and inelastic skeletal specific storage (SSV) coefficients.

\item \texttt{CONSTANT\_NODELAY\_THICKNESS}---keyword to indicate that the thickness and void ratio of no delay interbeds will not be varied during the simulation.

\item \texttt{CELL\_FRACTION}---keyword to indicate that the thickness of interbeds will be specified in terms of the fraction of cell thickness.

\item \texttt{COMPACTION}---keyword to specify that record corresponds to the compaction.

\item \texttt{FILEOUT}---keyword to specify that an output filename is expected next.

\item \texttt{compactionfile}---name of the binary output file to write compaction information.

\item \texttt{TS6}---keyword to specify that record corresponds to a time-series file.

\item \texttt{FILEIN}---keyword to specify that an input filename is expected next.

\item \texttt{ts6\_filename}---defines a time-series file defining time series that can be used to assign time-varying values. See the ``Time-Variable Input'' section for instructions on using the time-series capability.

\item \texttt{OBS6}---keyword to specify that record corresponds to an observations file.

\item \texttt{obs6\_filename}---name of input file to define observations for the IBC package. See the ``Observation utility'' section for instructions for preparing observation input files. Table \ref{table:obstype} lists observation type(s) supported by the IBC package.

\end{description}
\item \textbf{Block: DIMENSIONS}

\begin{description}
\item \texttt{nibccells}---is the number of IBC cells.  More than 1 IBC cell can be assigned to a GWF cell; however, only 1 GWF cell can be assigned to a single IBC cell.

\end{description}
\item \textbf{Block: GRIDDATA}

\begin{description}
\item \texttt{ske\_cr}---is elastic skeletal specific storage.

\item \texttt{sgm}---is specific gravity of moist or unsaturated sediments.

\item \texttt{sgs}---is specific gravity of saturated sediments.

\end{description}
\item \textbf{Block: PACKAGEDATA}

\begin{description}
\item \texttt{ibcno}---integer value that defines the IBC interbed number associated with the specified PACKAGEDATA data on the line. IBCNO must be greater than zero and less than or equal to NIBCCELLS.  IBC information must be specified for every IBC cell or the program will terminate with an error.  The program will also terminate with an error if information for a IBC intebed number is specified more than once.

\item \texttt{cellid}---is the cell identifier, and depends on the type of grid that is used for the simulation.  For a structured grid that uses the DIS input file, CELLID is the layer, row, and column.   For a grid that uses the DISV input file, CELLID is the layer and CELL2D number.  If the model uses the unstructured discretization (DISU) input file, CELLID is the node number for the cell.

\item \texttt{cdelay}---character string that defines the subsidence delay type for the IBC cell. Possible subsidence package CDELAY strings include: NODELAY--character keyword to indicate that delay will not be simulated in the IBC cell.  DELAY--character keyword to indicate that delay will be simulated in the IBC cell.

\item \texttt{pcs0}---is the initial preconsolidation stress or initial offset from the calculated initial effective stress in the IBC interbed, in units of height of a column of water. PCS0 is the initial offset from the calculated initial effective stress if INTERBED\_STRESS\_OFFSET is specified in the OPTIONS block.

\item \texttt{thick\_frac}---is the interbed thickness or cell fraction of the IBC interbed. Interbed thickness is specified as a fraction of the cell thickness if CELL\_FRACTION is specified in the OPTIONS block.

\item \texttt{rnb}---is the interbed material factor equivalent number of interbeds for the system of delay interbeds for interbed IBCNO. RNB must be greater than or equal to 1 if CDELAY is DELAY. Otherwise, RNB can be any value.

\item \texttt{ssv\_cc}---is the initial inelastic skeletal specific storage or compression index in the IBC interbed. The initial inelastic skeletal specific storage is specified if STORAGE\_COEFFICIENT is specified in the OPTIONS block.

\item \texttt{sse\_cr}---is the initial elastic skeletal specific storage or recompression index in the IBC interbed. The initial elastic skeletal specific storage is specified if STORAGE\_COEFFICIENT is specified in the OPTIONS block.

\item \texttt{theta}---is the initial porosity of the IBC interbed.

\item \texttt{kv}---is the vertical hydraulic conductivity of the IBC interbed. KV must be greater than 0 if CDELAY is DELAY. Otherwise, KV can be any value.

\item \texttt{h0}---is the initial head in the IBC interbed. H0 can be anyvalue if CDELAY is NODELAY.

\item \texttt{boundname}---name of the IBC cell cell.  BOUNDNAME is an ASCII character variable that can contain as many as 40 characters.  If BOUNDNAME contains spaces in it, then the entire name must be enclosed within single quotes.

\end{description}
\item \textbf{Block: PERIOD}

\begin{description}
\item \texttt{iper}---integer value specifying the starting stress period number for which the data specified in the PERIOD block apply.  IPER must be less than or equal to NPER in the TDIS Package and greater than zero.  The IPER value assigned to a stress period block must be greater than the IPER value assigned for the previous PERIOD block.  The information specified in the PERIOD block will continue to apply for all subsequent stress periods, unless the program encounters another PERIOD block.

\item \texttt{sig0}---is the stress offset for the cell. SIG0 is added to the calculated geostatic stress for the cell. SIG0 is specified only if GEO\_STRESS\_OFFSET is specified in the OPTIONS block.

\end{description}

