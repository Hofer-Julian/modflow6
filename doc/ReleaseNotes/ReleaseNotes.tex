\documentclass[11pt,twoside,twocolumn]{usgsreport}
\usepackage{usgsfonts}
\usepackage{usgsgeo}
\usepackage{usgsidx}
\usepackage[tabletoc]{usgsreporta}

\usepackage{amsmath}
\usepackage{algorithm}
\usepackage{algpseudocode}
\usepackage{bm}
\usepackage{calc}
\usepackage{natbib}
\usepackage{graphicx}
\usepackage{longtable}

\makeindex
\usepackage{setspace}
% uncomment to make double space 
%\doublespacing
\usepackage{etoolbox}
%\usepackage{verbatim}

\usepackage{titlesec}

\usepackage[hidelinks]{hyperref}
\hypersetup{
    pdftitle={MODFLOW 6 Release Notes},
    pdfauthor={MODFLOW 6 Development Team},
    pdfsubject={MODFLOW 6 Release Notes},
    pdfkeywords={MODFLOW, groundwater model, simulation},
    bookmarksnumbered=true,     
    bookmarksopen=true,         
    bookmarksopenlevel=1,       
    colorlinks=true,
    allcolors={blue},          
    pdfstartview=Fit,           
    pdfpagemode=UseOutlines,
    pdfpagelayout=TwoPageRight
}


\graphicspath{{./Figures/}}
\newcommand{\modflowversion}{mf6.0.2.117}
\newcommand{\modflowdate}{August 28, 2018}
\newcommand{\currentmodflowversion}{Version \modflowversion---\modflowdate}



\renewcommand{\cooperator}
{the \textusgs\ Water Availability and Use Science Program}
\renewcommand{\reporttitle}
{MODFLOW 6 Release Notes}
\renewcommand{\coverphoto}{coverimage.jpg}
\renewcommand{\GSphotocredit}{Binary computer code illustration.}
\renewcommand{\reportseries}{}
\renewcommand{\reportnumber}{}
\renewcommand{\reportyear}{2017}
\ifdef{\reportversion}{\renewcommand{\reportversion}{\currentmodflowversion}}{}
\renewcommand{\theauthors}{MODFLOW 6 Development Team}
\renewcommand{\thetitlepageauthors}{\theauthors}
%\renewcommand{\theauthorslastfirst}{}
\renewcommand{\reportcitingtheauthors}{MODFLOW 6 Development Team}
\renewcommand{\colophonmoreinfo}{}
\renewcommand{\theoffice}{Office of Groundwater \\ U.S. Geological Survey \\ Mail Stop 411 \\ 12201 Sunrise Valley Drive \\ Reston, VA 20192 \\ (703) 648-5001}
\renewcommand{\reportbodypages}{}
\urlstyle{rm}
\renewcommand{\reportwebsiteroot}{https://doi.org/10.5066/}
\renewcommand{\reportwebsiteremainder}{F76Q1VQV}
%\renewcommand{\doisecretary}{RYAN K. ZINKE}
%\renewcommand{\usgsdirector}{William H. Werkheiser}
%\ifdef{\usgsdirectortitle}{\renewcommand{\usgsdirectortitle}{Acting Director}}{}
\ifdef{\usgsissn}{\renewcommand{\usgsissn}{}}{}
\renewcommand{\theconventions}{}
\definecolor{coverbar}{RGB}{32, 18, 88}
\renewcommand{\bannercolor}{\color{coverbar}}
%\renewcommand{\thePSC}{the MODFLOW 6 Development Team}
%\renewcommand{\theeditor}{Christian D. Langevin}
%\renewcommand{\theillustrator}{None}
%\renewcommand{\thefirsttypesetter}{Joseph D. Hughes}
%\renewcommand{\thesecondtypesetter}{Cian Dawson}
\renewcommand{\reportrefname}{References Cited}

\newcommand{\customcolophon}{
Publishing support provided by the U.S. Geological Survey \\
\theauthors
\newline \newline
For information concerning this publication, please contact:
\newline \newline
Office of Groundwater \\ U.S. Geological Survey \\ Mail Stop 411 \\ 12201 Sunrise Valley Drive \\ Reston, VA 20192 \\ (703) 648--5001 \\
https://water.usgs.gov/ogw/
}


\begin{document}

%\makefrontcover
\ifdef{\makefrontcoveralt}{\makefrontcoveralt}{\makefrontcover}

%\makefrontmatter
%\maketoc
\ifdef{\makefrontmatterabv}{\makefrontmatterabv}{\makefrontmatter}

\onecolumn
\pagestyle{body}
\RaggedRight
\hbadness=10000
\pagestyle{body}
\setlength{\parindent}{1.5pc}

% -------------------------------------------------
\section{Introduction}
This document describes MODFLOW 6 Version \modflowversion.  This distribution is packaged for personal computers using the Microsoft Windows 7 operating system, although it may run on other versions of Windows.  The executable file was compiled for 64-bit Windows operating systems and should run on most personal computers.

Version numbers for MODFLOW 6 will follow a major.minor.revision format.  The major number will be increased when there are substantial new changes that may break backward compatibility.  The minor number will be increased when important, but relatively minor new functionality is added.  The revision number will be added when errors are corrected in either the program or input files.

% -------------------------------------------------
\section{History}
This section describes changes introduced into MODFLOW 6 with each release.  These changes may substantially affect users.

\begin{itemize}

\item \currentmodflowversion

\underline{STRESS PACKAGES}
\begin{itemize}
\item If the AUXMULTNAME keyword was used in combination with time series, then the multiplier was erroneously applied to all time series, and not just the time series in the column to be scaled.  
\end{itemize}

\underline{SOLUTION}
\begin{itemize}
\item Fixed bug related to not allocating the preconditioner work array if a non-zero drop tolerance is specified but the number of levels is not specified or specified to be zero. In the case where the number of levels is not specified or specified to be zero the preconditioner work array is dimensioned to the product of the number of cells (NEQ) and the maximum number of connections for any cell.
\item Updated linear solver output so number of levels and drop tolerance are output if either are specified to be greater than zero. 
\end{itemize}

\item Version mf6.0.2--Feb 23, 2018

\underline{BASIC FUNCTIONALITY}
\begin{itemize}
\item Added a new option, called SAVE\_SPECIFIC\_DISCHARGE to the Node Property Flow Package.  When invoked, $x$, $y$, and $z$ specific storage components are calculated for the center of each model cell and written to the binary budget file.
\item For binary input of grid data, such as initial heads, the array reading utility was not reading a header record consisting of KSTP, KPER, PERTIM, TOTIM, TEXT, NLAY, NROW, NCOL.  This meant that a binary head file written by MODFLOW could not be used as input for a subsequent simulation.  For binary input, the array reading utility now reads a header record before reading the array values.
\item The NOGRB option in the discretization packages was not working.  This option will now prevent the binary grid file from being written.
\item Removed the PRIVATE attribute for two methods of the discretization packages so that the program works as intended with the latest Intel Fortran release.
\item Switched to using a long integer for the memory manager so that memory usage is calculated correctly for large models.
\end{itemize}

\underline{STRESS PACKAGES}
\begin{itemize}
\item If a steady-state stress period followed a transient stress period, the storage terms written to the budget file were not being reset to zero.  The program now initializes these budget values to zero for steady-state periods before they are written.
\end{itemize}

\underline{ADVANCED STRESS PACKAGES}
\begin{itemize}
\item The STATUS INACTIVE option was not working correctly for the MAW Package.
\item Modified the MAW connection conductance calculation so that a linear relation between the water level in a cell and saturation is used for the standard formulation. In the previous version, the same quadratic saturation function was being used for the standard and Newton-Raphson formulation to calculate the MAW connection conductance. 
\item Modified the MAW Package so that the top and bottom of the screen for a connection are reset to the top and bottom of the cell, respectively, for SPECIFIED, THEIM, SKIN, and CUMULATIVE conductance equations (CONDEQN). Also, the program will now terminate with an error if a MAW well using SPECIFIED, THEIM, SKIN, or CUMULATIVE conductance equations has more than one connection to a single GWF cell. 
\item Modified the MAW package so that the well bottom (BOTTOM) is reset to the cell bottom in the lowermost GWF cell connection in cases where the specified well bottom is above the bottom of this GWF cell.
\end{itemize}

\underline{SOLUTION}
\begin{itemize}
\item Prior to applying pseudo transient continuation terms, the Iterative Model Solution confirms that the L2-norm exceeds the previous L2-norm.  If it doesn't then pseudo transient continuation is turned off.  This fixes a rare situation in which convergence could not be achieved for consecutive steady state solutions with the same or similar answers. 
\end{itemize}


\item Version mf6.0.1--Sep 28, 2017

\underline{BASIC FUNCTIONALITY}
\begin{itemize}
\item There is no requirement that FTYPE entries in the GWF name file should be upper case; however, an upper case convention was being enforced.  FTYPE entries can now be specified using any case.
\item Tab characters within model input files were not being skipped correctly.  This has been fixed.
\item The program was updated to use the ``approved for release'' disclaimer.  The previous version was still using a ``preliminary software'' disclaimer.
\item The source code for time series and time array series was refactored.  Included in the refactoring was a correction to time array series to allow the time array to change from one stress period to the next.  The source file TimeSeriesGroupList.f90 was renamed to TimeSeriesFileList.f90.
\end{itemize}

\underline{STRESS PACKAGES}
\begin{itemize}
\item Fixed inconsistency with CHD package observation name in code (\texttt{chd-flow}) and name in the input-output document (\texttt{chd}). Using name defined in input-output document (\texttt{chd}).
\item The cell area was not being used in the calculation of recharge and evapotranspiration when list input was used with time series.
\item The AUXMULTNAME option was not being applied for recharge and evapotranspiration when the READASARRAYS option was used.
\item The program was not terminating with an error if a PERIOD block was encountered with an iper value equal to the previous iper value.  Program now terminates with an error.
\end{itemize}

\underline{ADVANCED STRESS PACKAGES}
\begin{itemize}
\item Fixed incorrect sign for SFR package exchange with GWF model (\texttt{sfr}).
\item Added option to specify \texttt{none} as the \texttt{bedleak} for a lake-\texttt{GWF} connection in lake (LAK) package. This option makes the lake-\texttt{GWF} connection conductance solely a function of aquifer properties in the connected \texttt{GWF} cell and lakebed sediments are assumed to be absent for this connection.
\item Fixed bug in lake (LAK) and multi-aquifer well (MAW) packages that only reset steady-state flag if lake and/or multi-aquifer data are read for a stress period (in the pak\_rp() routines). Using pointer to GWF iss variable in the LAK package and resetting the MAW steady state flag in maw\_rp() routine every stress period, regardless of whether MAW data are specified for a stress period.
\item Added a convergence check routine to the GWF Mover Package that requires at least two outer iterations if there are any active movers.  Because mover rates are lagged by one outer iteration, at least two outer iterations are required for some problems.
\item Changed the behavior of the LAK Package so that recharge and evapotranspiration are applied to a vertically connected GWF model cell if the lake status is INACTIVE.  Prior to this change, recharge and evapotranspiration were only applied to an underlying GWF model cell if the lake was dry.
\end{itemize}

\underline{SOLUTION}
\begin{itemize}
\item Fixed bug in IMS that allowed convergence when outer iteration HCLOSE value was satisfied but the model did not converge during the inner iterations.
\item Added STRICT rclose\_option that uses a infinity-Norm RCLOSE criteria but requires HCLOSE and RCLOSE be satisfied on the first inner iteration of an outer iteration. The STRICT option is identical to the closure criteria approach use in the PCG Package in MODFLOW-2005.
\end{itemize}

\underline{EXCHANGES}
\begin{itemize}
\item Use of an OPEN/CLOSE file was not being allowed for the OPTIONS and DIMENSIONS blocks of the GWF6-GWF6 exchange input file.  OPEN/CLOSE input is now allowed for both of these blocks.
\end{itemize}

\item
Version mf6.0.0---August 10, 2017

\underline{BASIC FUNCTIONALITY}
\begin{itemize}
\item Removed support for the SINGLE observation type.  All observations must be CONTINUOUS, which means observation values are written for every time step. 
\item Added support for a no-data value (3.0E30), which can be used as a placeholder in a time-series file containing multiple time series. Use of the no-data value facilitates combining separate time series into a single file when the time series contain records for differing simulation times.
\item Model names specified in the simulation name file cannot have spaces in them.  A check was implemented to terminate with an error if the model name contains spaces.  Model names cannot exceed 16 characters.  Trailing spaces are allowed.
\item The name and version of the compiler used to make the run file is now written to the terminal and to the simulation list file.
\item Many of the Fortran source files were modified and reformatted.  Unused variables were removed.
\end{itemize}

\underline{ADVANCED STRESS PACKAGES}
\begin{itemize}
\item Updated MAW package so that well connection conductance calculations correctly account for THICKSTRT in the NPF package for layers that use THICKSTRT (and are confined).
\item Added \texttt{CUMULATIVE} \texttt{coneqn} (conductance) option to MAW package.
\item Fixed bug in LAK package weir lake outlet calculation.
\item Fixed bug in LAK package when internal outlets were specified and combined with the MVR package that was also moving water internally in the same LAK package.
\item Updated the table created when PRINT\_FLOWS is specified in the LAK package OPTIONS block to include internal flow terms if NOUTLETS is greater than 0. 
\item Renamed Lake Tables DIMENSIONS block NENTRIES to NROW and added NCOL to DIMENSIONS block.
\item Eliminated MAXIMUM\_OUTLET\_DEPTH = 10 [L] as default behavior for MANNING and WEIR LAK package lake outlet types. The maximum depth threshold was used in MODFLOW-2005 lake package because a table was used to calculate lake outflows to SFR. Can still use maximum depth threshold in develop versions of MODFLOW 6 by specifying MAXIMUM\_OUTLET\_DEPTH in the options block with a value.
\item Removed MULTILAYER option for UZF package---this option didn't actually do anything.
\item Added the requirement that the UZF number be specified as the first value on each line in the PACKAGEDATA block.
\item Renamed MAXBOUND in the DIMENSIONS block of the SFR Package to be NREACHES.
\item Implemented a check in the SFR Package to make sure that information is specified in the PACKAGEDATA block for every reach.  Program terminates with an error if information for a reach is not found.
\end{itemize}

\item
Version mf6beta0.9.03---June 23, 2017

\underline{BASIC FUNCTIONALITY}
\begin{itemize}
\item Renamed all FTYPE keywords to version 6.  They were named with an 8.  So, for example, the GHB Package is now activated in the GWF name file using ``GHB6'' instead of ``GHB8''.
\item Keywords in the simulation name file must now be specified as TDIS6, GWF6, and GWF6-GWF6 to be consistent.
\item The DIS Package had grid offsets (XOFFSET and YOFFSET) that could be specified as options.  These offsets were relative to the upper-left corner of the model grid.  The default value for YOFFSET was set to the sum of DELR so that (0, 0) would correspond to the lower-left corner of the model grid.  These options have been removed and replaced with XORIGIN and YORIGIN, which is the coordinate of the lower-left corner of the model grid.  The default value is zero for XORIGIN and YORIGIN.
\item Can now specify XORIGIN, YORIGIN, and ANGROT as options for the DISV and DISU packages.  These values are written to the binary grid file, which can be used by post-processors to locate the model grid in space.  These options have no affect on the simulation results.  The default value is 0.0 if not specified.
\item Added a new option to the TDIS input file called START\_DATE\_TIME.  This is a 30 character string that represents the simulation starting date and time, preferably in the format described at https://www.w3.org/TR/NOTE-datetime.  The value provided by the user has no affect on the simulation, but if it is provided, the value is written to the simulation list file.
\item Changed default behavior for how memory usage is written to the end of the simulation list file.  Added new MEMORY\_PRINT\_OPTION to simulation options to control how memory usage is written.
\item Corrections were made to the memory manager to ensure that all memory is deallocated at the end of a simulation.
\end{itemize}

\underline{INTERNAL FLOW PACKAGES}
\begin{itemize}
\item Changed the way hydraulic conductivity is specified in the NPF Package.  Users no longer specify HK, VK, and HANI.  Hydraulic conductivity is now specified as ``K''.  If hydraulic conductivity is isotropic, then this is all that needs to be specified.  For anisotropic cases, the user can specify an optional ``K22'' array and an optional ``K33'' array.  For an unrotated conductivity ellipsoid ``K22'' corresponds to hydraulic conductivity in the y direction and ``K33'' corresponds to hydraulic conductivity in the z direction, respectively.
\end {itemize}

\underline{ADVANCED STRESS PACKAGES}
\begin{itemize}
\item Modified the MAW Package to include the effects of aquifer anisotropy in the calculation of conductance.
\item Simplified the SFR Package connectivity to reflect feedback from beta users. There is no longer a requirement to connect reaches that do not have flow between them.  Program will now terminate with an error if this condition is encountered.
\item Added simple routing option to SFR package. This is the equivalent of the specified depth option (icalc=0) in previous versions of MODFLOW. If water is available in the reach, then there can be leakage from the SFR reach into the aquifer.  If no water is available, then no leakage is applied.  STAGE keyword also added and only applies to reaches that use the simple routing option. If the STAGE keyword is not specified for reaches that use the simple routing option the specified stage is set to the top of the reach (depth = 0).
\item Added functionality to pass SFR leakage to the aquifer to the highest active layer.
\item Converted SFR Manning's to a time-varying, time series aware variable.  
\item Updated LAK package so that conductance calculations correctly account for THICKSTRT in the NPF package for layers that use THICKSTRT (and are confined). Also updated EMBEDDEDH and EMBEDDEDV so that the conductance for these connection types are constant for confined layers.
\item Converted UZF stress-period data to time series aware data.
\item Added time-series aware AUXILIARY variables to UZF package.
\item Implemented AUXMULTNAME in options block for UZF package (AUXILIARY variables have to be specified). AUXMULTNAME is applied to the GWF cell area and is used to simulated more than one UZF cell per GWF cell. This could be used to simulate different land use classifications (i.e., agricultural and natural land use types) in the same GWF cell.
\end{itemize}

\underline{SOLUTION}
\begin{itemize}
\item Reworked IMS convergence information so that model specific convergence information is also printed to each model listing file when PRINT\_OPTION ALL is specified in the IMS OPTIONS block.
\item Added csv output option for IMS convergence information. Solution convergence information and model specific convergence information (if the solution includes more than one model) is written to a comma separated value file. If PRINT\_OPTION is NONE or SUMMARY, csv output includes maximum head change convergence information at the end of each outer iteration for each time step. If PRINT\_OPTION is ALL, csv output includes maximum head change and maximum residual convergence information for the solution and each model (if the solution includes more than one model) and linear acceleration information for each inner iteration. 
\end{itemize}

\item
Version mf6beta0.9.02---May 19, 2017
\begin{itemize}
\item Renamed gwf3.f90 to be lower case.
\item Added the missing ``divrate'' variable to the ``sfrsetting'' description in mf6io.pdf.
\item Added additional error trapping to the array reading utilities.
\item There was a problem with the binary budget file when a GWF Exchange was used to connect a GWF Model with itself.  This has been fixed.
\item Standardized `\texttt{to-mvr}' cell-by-cell item in standard stress packages and UZF package.
\item Fixed incorrect `\texttt{UZF-EVT}' budget accumulator used in GWF listing budget. 
\item Standardized justification of cell-by-cell `\texttt{text}' strings.
\item Standardized use of AUXILIARY keyword.
\end{itemize}

\item
Version mf6beta0.9.01---May 11, 2017
\begin{itemize}
\item Added a copy of the third MODFLOW 6 report. 
\item Made several minor corrections to doc/mf6io.pdf.  
\item If vertices were specified for DISU, then the last header line was not written to the binary grid file.  This has been corrected.
\end{itemize}

\item
Version mf6beta0.9.00---May 10, 2017
\begin{itemize}
\item First public release of MODFLOW 6 in beta form. 
\end{itemize}
\end{itemize}


% -------------------------------------------------
\section{Known Issues}
This section describes known issues with this release of MODFLOW 6.

\begin{enumerate}

\item
The capability to use Unsaturated Zone Flow (UZF) routing beneath lakes and streams has not been implemented.

\end{enumerate}


% -------------------------------------------------
\section{Distribution File}
The following distribution file is for use on personal computers: \modflowversion.zip.  The distribution file is a compressed zip file. The following directory structure is incorporated in the zip file:

% folder structured created by python script
\begin{verbatim}
mf6.0.2.34/ 
    bin/ 
    doc/ 
    examples/ 
        ex01-twri/ 
        ex02-tidal/ 
        ex03-bcf2ss/ 
        ex04-fhb/ 
        ex05-mfusg1disu/ 
        ex06-mfusg1disv/ 
        ex07-mfusg1lgr/ 
        ex08-mfusg1xt3d/ 
        ex09-bump/ 
        ex10-bumpnr/ 
        ex11-disvmesh/ 
        ex12-hanicol/ 
        ex13-hanirow/ 
        ex14-hanixt3d/ 
        ex15-whirlsxt3d/ 
        ex16-mfnwt2/ 
        ex17-mfnwt3h/ 
        ex18-mfnwt3l/ 
        ex19-zaidel/ 
        ex20-keating/ 
        ex21-sfr1/ 
        ex22-lak2/ 
        ex23-lak4/ 
        ex24-neville/ 
        ex25-flowing-maw/ 
        ex26-Reilly-maw/ 
        ex27-advpakmvr/ 
        ex28-mflgr3/ 
        ex29-vilhelmsen-gc/ 
        ex30-vilhelmsen-gf/ 
        ex31-vilhelmsen-lgr/ 
        ex32-periodicbc/ 
    make/ 
    msvs/ 
    src/ 
        Exchange/ 
        Model/ 
            Geometry/ 
            GroundWaterFlow/ 
            ModelUtilities/ 
        Solution/ 
            SparseMatrixSolver/ 
        Timing/ 
        Utilities/ 
            Memory/ 
            Observation/ 
            OutputControl/ 
            TimeSeries/ 
    utils/ 
        mf5to6/ 
            make/ 
            msvs/ 
            src/ 
                LGR/ 
                MF2005/ 
                NWT/ 
                Preproc/ 
        zonebudget/ 
            make/ 
            msvs/ 
            src/ 
\end{verbatim}


It is recommended that no user files are kept in the \modflowversion~directory structure.  If you do plan to put your own files in the \modflowversion~directory structure, do so only by creating additional subdirectories.

% -------------------------------------------------
\section{Installation and Executation}
There is no installation of MODFLOW 6 other than the requirement that \modflowversion.zip must be unzipped into a location where it can be accessed.  

To make the executable versions of MODFLOW 6 accessible from any directory, the directory containing the executables should be included in the PATH environment variable.  Also, if a prior release of MODFLOW 6 is installed on your system, the directory containing the executables for the prior release should be removed from the PATH environment variable.

As an alternative, the executable file, mf6.exe, in the \modflowversion{}/bin directory can be copied into a directory already included in the PATH environment variable.

To run MODFLOW 6, simply type \texttt{mf6} in a terminal window.  The current working directory must be set to a location where the model input files are located.  Upon execution, MODFLOW 6 will immediately look for file with the name \texttt{mfsim.nam} in the current working directory, and will terminate with an error if it does not find this file.

% -------------------------------------------------
\section{Compiling MODFLOW 6}
MODFLOW 6 has been compiled using Intel Visual Fortran and gfortran on the Windows and Mac/OS operating systems.  Because the program uses relatively new Fortran capabilities, newer versions of the compilers may be required for successful compilation.  For example, to use gfortran to compile MODFLOW 6, gfortran version 4.9 or newer must be used.  If you have gfortran installed on your computer, you can tell which version it is by entering ``\verb|gfortran --version|'' at a terminal window.

This distribution contains the Microsoft Visual Studio project files for compiling MODFLOW 6 on Windows using the Intel Fortran Compiler.  The files have been used successfully with Visual Studio 2017 and Intel(R) Visual Fortran Compiler 18.0.1.156.

This distribution also comes with a makefile for compiling MODFLOW 6 with gfortran.  The makefile is contained in the \texttt{make} folder.

For those familiar with Python, the pymake package can also be used to compile MODFLOW 6.  Additional information on the Python pymake utility can be found at: https://github.com/modflowpy/pymake.  

% -------------------------------------------------
\section{System Requirements}
MODFLOW 6 is written in Fortran.  It uses features from the 95, 2003, and 2008 language.  The code has been used on UNIX-based computers and personal computers running various forms of the Microsoft Windows operating system.

% -------------------------------------------------
\section{Testing}
The examples distributed with MODFLOW 6 can be run by navigating to the examples folder and executing the "run.bat" batch files within each example folder.  Alternatively, there is a runall.bat batch file under the examples folder that will run all of the test problems.

% -------------------------------------------------
\section{MODFLOW 6 Documentation}
Details on the numerical methods and the underlying theory for MODFLOW 6 are described in the following reports:

\begin{itemize}
\item
Hughes, J.D., Langevin, C.D., and Banta, E.R., 2017, Documentation for the MODFLOW 6 framework: U.S. Geological Survey Techniques and Methods, book 6, chap. A57, 40 p., https://doi.org/10.3133/tm6A57.

\item
Langevin, C.D., Hughes, J.D., Banta, E.R., Niswonger, R.G., Panday, Sorab, and Provost, A.M., 2017, Documentation for the MODFLOW 6 Groundwater Flow Model: U.S. Geological Survey Techniques and Methods, book 6, chap. A55, 197 p., https://doi.org/10.3133/tm6A55.

\item
Provost, A.M., Langevin, C.D., and Hughes, J.D., 2017, Documentation for the ``XT3D'' option in the Node Property Flow (NPF) Package of MODFLOW 6: U.S. Geological Survey Techniques and Methods, book 6, chap. A56, 40 p., https://doi.org/10.3133/tm6A56.

\end{itemize}
 
\noindent Description of the MODFLOW 6 input and output is included in this distribution in the ``doc'' folder as mf6io.pdf.

% -------------------------------------------------
\section{Test Problems}
The following is a list of test problems distributed with MODFLOW 6.  Characteristics of these tests are contained in Table \ref{table:examples}.


% example tex files created by python scripts

\begin{itemize}
\item ex01-twri---This is the TWRI problem described in the MODFLOW-2005 documentation and included with the MODFLOW-2005 examples. 
\item ex02-tidal---This problem demonstrates the time series and observation capabilities of MODFLOW 6.  Use of multiple boundary packages for a single simulation is also demonstrated by including three recharge packages.
\item ex03-bcf2ss---This is the BCF2SS problem that is distributed with MODFLOW-2005.  This problem demonstrates the wetting and drying capability in MODFLOW 6. The MODFLOW 6 problem is constructed with two layers (like the MODFLOW-2005 model) but the thickness of the confining bed is included in model layer 2 and the horizontal hydraulic conductivity of layer 2 is half that of model layer 2 in the MODFLOW-2005 model in order to calculate the correct horizontal conductance.
\item ex04-fhb---This problem is included with the MODFLOW-2005 examples. This problem demonstrates how the time-series functionality, combined with the Constant-Head and Well Packages, can be used to replace the Flow and Head Boundary (FHB) Package.
\item ex05-mfusg1disu---This is the first test problem presented in the MODFLOW-USG documentation.  It is included as an example problem to demonstrate a simple unstructured groundwater flow model.  The model uses ghost nodes to improve the accuracy of the groundwater flow solution.
\item ex06-mfusg1disv---This is the first test problem presented in the MODFLOW-USG documentation.  It is included as an example problem to demonstrate the DISV Package for a simple groundwater flow model.  The model uses ghost nodes to improve the accuracy of the groundwater flow solution.
\item ex07-mfusg1lgr---This is also the first test problem presented in the MODFLOW-USG manual; however, it is represented using two separate structured models.  The models are connected using a Groundwater Flow to Groundwater Flow (GWF-GWF) Exchange.  These two models are solved simultaneously in the same matrix equations.  A ghost-node correction is also applied to improve the flow calculation between models.
\item ex08-mfusg1xt3d---This is the first test problem presented in the MODFLOW-USG documentation.  It is included as an example problem to demonstrate the DISV Package for a simple groundwater flow model.  The model uses the XT3D formulation to improve the accuracy of the groundwater flow solution.
\item ex09-bump---This is a one-layer steady-state problem involving wetting and drying.  There is a rise in the bottom surface of the model, and groundwwater flows around the rise.
\item ex10-bumpnr---This is a one-layer steady-state problem designed to test the Newton-Raphson approach.  There is a rise in the bottom surface of the model, and groundwwater flows around the rise.
\item ex11-disvmesh---Demonstration of a triangular mesh with the DISV Package to discretize a circular island with a radius of 1500 meters.  The model has 2 layers and uses 2778 vertices (NVERT) to delineate 5240 cells per layer (NCPL).  General-head boundaries are assigned to model layer 1 for cells outside of a 1025 m radius circle.  Recharge is applied to the top of the model. 

\item ex12-hanicol---Simple steady state model using a regular MODFLOW grid to simulate the response of an anisotropic confined aquifer to a pumping well. A constant-head boundary condition surrounds the active domain.  K22 is set to 100.0, which causes hydraulic conductivity in column direction to be 100 x more than K, which is in row direction.  Drawdown is more pronounced in column direction.

\item ex13-hanirow---Simple steady state model using a regular MODFLOW grid to simulate the response of an anisotropic confined aquifer to a pumping well. A constant-head boundary condition surrounds the active domain.  K22 is set to 0.01, which causes K in column direction to be 100 x less than K in the row direction.  Drawdown is more pronounced in row direction.

\item ex14-hanixt3d---Simple steady state model using a regular MODFLOW grid to simulate the response of an anisotropic confined aquifer to a pumping well. For this problem, the XT3D formulation is used so that hydraulic conductivity ellipse can be rotated in the x-y plane.  A constant-head boundary condition surrounds the active domain.  K22 is set to 0.01, which causes hydraulic conductivity in the column direction (prior to rotation) to be 100 x less than K in the row direction.  This ellipse is then rotated in the x-y plane by specifying a value for ANGLE1 in the NPF Package.  ANGLE1 is specifed with a constant value of 15 degrees for the entire grid, which means the dominant K component is rotated 15 degrees counter clockwise.  Drawdown is more pronounced along the dominant axis of the hydraulic conductivity ellipse.

\item ex15-whirlsxt3d---This is a 10 layer steady-state problem involving anisotropic groundwater flow.  The XT3D formulation is used to represent variable hydraulic conductivitity ellipsoid orientations.  The resulting flow pattern consists of groundwater whirls, as described in the XT3D documentation report.
\item ex16-mfnwt2---This is the the second example problem described in the MODFLOW-NWT documentation (Niswonger and others, 2011) and is based on ``problem 2'' in McDonald and others (1991). A fourth steady-state stress period has been added to the problem for comparison with fig. 8D in Niswonger and others(2011).
\item ex17-mfnwt3h---This is the high recharge case of the third example problem described in the MODFLOW-NWT documentation.
\item ex18-mfnwt3l---This is the low recharge case of the third example problem described in the MODFLOW-NWT documentation.
\item ex19-zaidel---This is the stair-step problem described in Zaidel (2013).  In this simulation, the Newton-Raphson formulation is used to improve simulation convergence.
\item ex20-keating---This is an example problem described in Keating and Zyvoloski (2009).  The problem involves recharge through the unsaturated zone onto an aquitard.  The Newton-Raphson formulation is used for this problem to obtain a solution.
\item ex21-sfr1---This is the stream-aquifer interaction example problem (test 1) from the Streamflow Routing Package documentation (Prudic and others, 1989).  The specified depth segments in the original problem have been converted to active reaches and the diversion has been converted from UPTO to FRACTION CPRIOR type. This problem is simulated using the Streamflow Routing (SFR) Package in MODFLOW 6.
\item ex22-lak2---This is the lake-stream-aquifer interaction example problem (test simulation 2) from the Lake Package documentation (Merritt and Konikow, 2000).  This problem is simulated using the Lake (LAK) and Streamflow Routing (SFR) Packages in MODFLOW 6. The Mover (MVR) Package is also used to exchange water between the SFR and LAK Packages.
\item ex23-lak4---This is the lake-aquifer interaction example problem (test simulation 4) from the Lake Package documentation (Merritt and Konikow, 2000).  This problem is simulated using the Lake (LAK) Package in MODFLOW 6.
\item ex24-neville---This is the multi-aquifer well simulation described in Neville and Tonkin (2004).  This problem is simulated using the Multi-Aquifer Well (MAW) Package in MODFLOW 6.
\item ex25-flowing-maw---This is a multi-aquifer well simulation that demonstrates how to implement the flowing well option available in Multi-Aquifer Well (MAW) Package in MODFLOW 6. Aquifer properties and initial heads are identical to Neville and Tonkin (2004).  The pumping rate for well in the center of the domain is 0.0 cubic meters per day and the flowing well discharge elevation and conductance are specified to be 0.0 meters and 7,500 square meters per day.
\item ex26-Reilly-maw---This is the unstressed multi-aquifer well simulation described in Reilly and others (1989).  This problem is simulated using the Multi-Aquifer Well (MAW) Package in MODFLOW 6.
\item ex27-advpakmvr---This is a variant of the unsaturated zone-lake-stream-aquifer interaction example problem (test simulation 2) from the Unsaturated Zone Flow Package documentation (Niswonger and others, 2006) that includes a two layer aquifer and two lakes connected to the stream network.  This problem is simulated using the Unsaturated Zone Flow (UZF), Lake (LAK), and Streamflow Routing (SFR) Packages in MODFLOW 6. The Mover (MVR) Package is also used to exchange water between the UZF, LAK, and SFR Packages. Infiltration rates, ET rates, streamed Ks and lakebed leakances were changed to lower the water table below the interface of layers 1 and 2. This was done to demonstrate unsaturated flow through multiple layers. Aquifer K values were also changed.
\item ex28-mflgr3---The is Example 3 from the MODFLOW-LGR2 documentation.
\item ex29-vilhelmsen-gc---This is the Globally Coarse (GC) model described in Vilhelmsen et al. (2012).
\item ex30-vilhelmsen-gf---This is the Globally Fine (GF) model described in Vilhelmsen et al. (2012).
\item ex31-vilhelmsen-lgr---This is the Local Grid Refinement (LGR) model described in Vilhelmsen et al. (2012).
\item ex32-periodicbc---Periodic boundary condition problem is based on Laattoe and others (2014). A MODFLOW 6 GWF-GWF Exchange is used to connect the left column with the right column. 
\end{itemize}



\small
\begin{longtable}{p{3cm} p{1cm} p{3cm} p{2.5cm}p{4cm}}
\caption{List of example problems and simulation characteristics}\tabularnewline


\hline
\hline
\textbf{Name} & \textbf{NPER} & \textbf{Namefile(s)} & \textbf{Dimensions (NLAY, NROW, NCOL), (NLAY, NCPL) or (NODES)}  & \textbf{Stress Packages} \\
\hline
\endfirsthead

\hline
\hline
\textbf{Name} & \textbf{NPER} & \textbf{Namefile(s)} & \textbf{Dimensions (NLAY, NROW, NCOL) or (NODES)}  & \textbf{Stress Packages} \\
\hline
\endhead


ex01-twri & 1 & \parbox[t]{3cm}{ twri.nam \\}& \parbox[t]{3cm}{ (3, 15, 15) \\}& \parbox[t]{4cm}{ CHD WEL DRN RCH  \\}\\
\hline
ex02-tidal & 4 & \parbox[t]{3cm}{ AdvGW\_tidal.nam \\}& \parbox[t]{3cm}{ (3, 15, 10) \\}& \parbox[t]{4cm}{ WEL RIV RCH GHB EVT  \\}\\
\hline
ex03-bcf2ss & 2 & \parbox[t]{3cm}{ bcf2ss.nam \\}& \parbox[t]{3cm}{ (2, 10, 15) \\}& \parbox[t]{4cm}{ WEL RIV RCH  \\}\\
\hline
ex04-fhb & 3 & \parbox[t]{3cm}{ fhb2015.nam \\}& \parbox[t]{3cm}{ (1, 3, 10) \\}& \parbox[t]{4cm}{ CHD WEL  \\}\\
\hline
ex05-mfusg1disu & 1 & \parbox[t]{3cm}{ flow.nam \\}& \parbox[t]{3cm}{ (121,) \\}& \parbox[t]{4cm}{ CHD  \\}\\
\hline
ex06-mfusg1disv & 1 & \parbox[t]{3cm}{ flow.nam \\}& \parbox[t]{3cm}{ (1, 121) \\}& \parbox[t]{4cm}{ CHD RCH  \\}\\
\hline
ex07-mfusg1lgr & 1 & \parbox[t]{3cm}{ model1.nam \\ model2.nam \\}& \parbox[t]{3cm}{ (1, 7, 7) \\ (1, 9, 9) \\}& \parbox[t]{4cm}{ CHD  \\ none \\}\\
\hline
ex08-mfusg1xt3d & 1 & \parbox[t]{3cm}{ flow.nam \\}& \parbox[t]{3cm}{ (1, 121) \\}& \parbox[t]{4cm}{ CHD RCH  \\}\\
\hline
ex09-bump & 1 & \parbox[t]{3cm}{ flowdivert.nam \\}& \parbox[t]{3cm}{ (1, 51, 51) \\}& \parbox[t]{4cm}{ CHD  \\}\\
\hline
ex10-bumpnr & 1 & \parbox[t]{3cm}{ flowdivert.nam \\}& \parbox[t]{3cm}{ (1, 51, 51) \\}& \parbox[t]{4cm}{ CHD  \\}\\
\hline
ex11-disvmesh & 1 & \parbox[t]{3cm}{ ci.nam \\}& \parbox[t]{3cm}{ (2, 5240) \\}& \parbox[t]{4cm}{ GHB RCH  \\}\\
\hline
ex12-hanicol & 1 & \parbox[t]{3cm}{ model.nam \\}& \parbox[t]{3cm}{ (1, 51, 51) \\}& \parbox[t]{4cm}{ CHD WEL  \\}\\
\hline
ex13-hanirow & 1 & \parbox[t]{3cm}{ model.nam \\}& \parbox[t]{3cm}{ (1, 51, 51) \\}& \parbox[t]{4cm}{ CHD WEL  \\}\\
\hline
ex14-hanixt3d & 1 & \parbox[t]{3cm}{ model.nam \\}& \parbox[t]{3cm}{ (1, 51, 51) \\}& \parbox[t]{4cm}{ CHD WEL  \\}\\
\hline
ex15-whirlsxt3d & 1 & \parbox[t]{3cm}{ model.nam \\}& \parbox[t]{3cm}{ (10, 10, 51) \\}& \parbox[t]{4cm}{ CHD WEL  \\}\\
\hline
ex16-mfnwt2 & 4 & \parbox[t]{3cm}{ test034\_nwtp2.nam \\}& \parbox[t]{3cm}{ (14, 40, 40) \\}& \parbox[t]{4cm}{ CHD RCH  \\}\\
\hline
ex17-mfnwt3h & 1 & \parbox[t]{3cm}{ nwtp3.nam \\}& \parbox[t]{3cm}{ (1, 80, 80) \\}& \parbox[t]{4cm}{ CHD RCH  \\}\\
\hline
ex18-mfnwt3l & 1 & \parbox[t]{3cm}{ nwtp3.nam \\}& \parbox[t]{3cm}{ (1, 80, 80) \\}& \parbox[t]{4cm}{ CHD RCH  \\}\\
\hline
ex19-zaidel & 1 & \parbox[t]{3cm}{ zaidel5m.nam \\}& \parbox[t]{3cm}{ (1, 1, 200) \\}& \parbox[t]{4cm}{ CHD  \\}\\
\hline
ex20-keating & 1 & \parbox[t]{3cm}{ keating.nam \\}& \parbox[t]{3cm}{ (80, 1, 400) \\}& \parbox[t]{4cm}{ RCH CHD  \\}\\
\hline
ex21-sfr1 & 2 & \parbox[t]{3cm}{ test1tr.nam \\}& \parbox[t]{3cm}{ (1, 15, 10) \\}& \parbox[t]{4cm}{ WEL EVT RCH GHB SFR  \\}\\
\hline
ex22-lak2 & 1 & \parbox[t]{3cm}{ lakeex2a.nam \\}& \parbox[t]{3cm}{ (5, 27, 17) \\}& \parbox[t]{4cm}{ EVT RCH SFR LAK CHD MVR  \\}\\
\hline
ex23-lak4 & 1 & \parbox[t]{3cm}{ lakeex4.nam \\}& \parbox[t]{3cm}{ (8, 36, 23) \\}& \parbox[t]{4cm}{ CHD RCH LAK  \\}\\
\hline
ex24-neville & 1 & \parbox[t]{3cm}{ NT\_Transient.nam \\}& \parbox[t]{3cm}{ (2, 101, 101) \\}& \parbox[t]{4cm}{ MAW  \\}\\
\hline
ex25-flowing-maw & 1 & \parbox[t]{3cm}{ FW\_Transient.nam \\}& \parbox[t]{3cm}{ (2, 101, 101) \\}& \parbox[t]{4cm}{ MAW  \\}\\
\hline
ex26-Reilly-maw & 1 & \parbox[t]{3cm}{ Reilly.nam \\}& \parbox[t]{3cm}{ (41, 16, 27) \\}& \parbox[t]{4cm}{ CHD MAW RCH  \\}\\
\hline
ex27-advpakmvr & 24 & \parbox[t]{3cm}{ uzfp3\_lakmvr\_v2.nam \\}& \parbox[t]{3cm}{ (2, 15, 10) \\}& \parbox[t]{4cm}{ SFR LAK WEL GHB UZF MVR  \\}\\
\hline
ex28-mflgr3 & 1 & \parbox[t]{3cm}{ ex3\_parent.nam \\ ex3\_child.nam \\}& \parbox[t]{3cm}{ (3, 15, 15) \\ (6, 15, 18) \\}& \parbox[t]{4cm}{ RIV CHD  \\ RIV  \\}\\
\hline
ex29-vilhelmsen-gc & 1 & \parbox[t]{3cm}{ parent.nam \\}& \parbox[t]{3cm}{ (9, 61, 49) \\}& \parbox[t]{4cm}{ RIV RCH  \\}\\
\hline
ex30-vilhelmsen-gf & 1 & \parbox[t]{3cm}{ TM9\_global\_gv.nam \\}& \parbox[t]{3cm}{ (25, 183, 147) \\}& \parbox[t]{4cm}{ RIV RCH  \\}\\
\hline
ex31-vilhelmsen-lgr & 1 & \parbox[t]{3cm}{ TM9\_parent\_GN.nam \\ Child\_GN.nam \\}& \parbox[t]{3cm}{ (9, 61, 49) \\ (25, 90, 78) \\}& \parbox[t]{4cm}{ RIV RCH  \\ RIV RCH  \\}\\
\hline
ex32-periodicbc & 1 & \parbox[t]{3cm}{ pbc.nam \\}& \parbox[t]{3cm}{ (190, 1, 100) \\}& \parbox[t]{4cm}{ CHD  \\}\\
\hline
\hline
\end{longtable}
\label{table:examples}
\normalsize



% -------------------------------------------------
\section{Disclaimer and Notices}

This software has been approved for release by the U.S. Geological Survey (USGS). Although the software has been subjected to rigorous review, the USGS reserves the right to update the software as needed pursuant to further analysis and review. No warranty, expressed or implied, is made by the USGS or the U.S. Government as to the functionality of the software and related material nor shall the fact of release constitute any such warranty. Furthermore, the software is released on condition that neither the USGS nor the U.S. Government shall be held liable for any damages resulting from its authorized or unauthorized use. Also refer to the USGS Water Resources Software User Rights Notice for complete use, copyright, and distribution information.

Notices related to this software are as follows:
\begin{itemize}
\item This software is a product of the U.S. Geological Survey, which is part of the U.S. Government.

\item This software is freely distributed. There is no fee to download and (or) use this software.

\item Users do not need a license or permission from the USGS to use this software. Users can download and install as many copies of the software as they need.

\item As a work of the United States Government, this USGS product is in the public domain within the United States. You can copy, modify, distribute, and perform the work, even for commercial purposes, all without asking permission. Additionally, USGS waives copyright and related rights in the work worldwide through CC0 1.0 Universal Public Domain Dedication (https://creativecommons.org/publicdomain/zero/1.0/).
\end{itemize}

% -------------------------------------------------
\section{References Cited}

\begin{itemize}
\item Keating, E., and Zyvoloski, G. 2009. A stable and efficient numerical algorithm for unconfined aquifer analysis. Ground Water, 47: 569--579. doi:10.1111/j.1745-6584.2009.00555.x

\item Laattoe, T., Post, V. E.A. and Werner, A. D. 2014, Spatial periodic boundary condition for MODFLOW. Groundwater, v. 52: 606--612. doi:10.1111/gwat.12086

\item Merritt, M.L., and Konikow, L.F. 2000, Documentation of a computer program to simulate lake-aquifer interaction using the MODFLOW ground-water flow model and the MOC3D solute-transport model. U.S. Geological Survey Water-Resources Investigations Report 00--4167, 146 p. https://pubs.er.usgs.gov/publication/wri004167.

\item Neville, C.J., and M.J. Tonkin. 2004. Modeling multiaquifer wells with MODFLOW.  Ground Water, 42: 910--919. doi:10.1111/j.1745-6584.2004.t01-9-.x

\item Panday, Sorab, Langevin, C.D., Niswonger, R.G., Ibaraki, Motomu, and Hughes, J.D. 2013, MODFLOW-USG version 1: An unstructured grid version of MODFLOW for simulating groundwater flow and tightly coupled processes using a control volume finite-difference formulation. U.S. Geological Survey Techniques and Methods, book 6, chap. A45, 66 p., https://pubs.usgs.gov/tm/06/a45.

\item Prudic, D.E. 1989, Documentation of a computer program to simulate stream-aquifer relations using a modular, finite-difference, ground-water flow model. U.S. Geological Survey Open-File Report 88--729, 113 p. https://pubs.er.usgs.gov/publication/ofr88729.

\item Reilly, T.E., O.L. Franke, and Bennett, G.D. 1989. Bias in groundwater samples caused by wellbore flow. Journal of Hydraulic Engineering 115, no. 2: 270--276. https://doi.org/10.1061/(ASCE)0733-9429(1989)115:2(270)

\item Vilhelmsen, T.N., Christensen, S., and Mehl, S.W., 2012, Evaluation of MODFLOW-LGR in connection with a synthetic regional-scale model. Ground Water, 50: 118--132. doi:10.1111/j.1745-6584.2011.00826.x

\item Zaidel, J. 2013, Discontinuous Steady-State Analytical Solutions of the Boussinesq Equation and Their Numerical Representation by MODFLOW. Groundwater, 51: 952--959. doi:10.1111/gwat.12019
\end{itemize}


\justifying
\vspace*{\fill}
\clearpage
\pagestyle{backofreport}
\makebackcover
\end{document}  