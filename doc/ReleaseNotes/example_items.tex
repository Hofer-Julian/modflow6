\begin{itemize}
\item ex01-twri---This is the TWRI problem described in the MODFLOW-2005 documentation and included with the MODFLOW-2005 examples. 
\item ex02-tidal---This problem demonstrates the time series and observation capabilities of MODFLOW 6.  Use of multiple boundary packages for a single simulation is also demonstrated by including three recharge packages.
\item ex03-bcf2ss---This is the BCF2SS problem that is distributed with MODFLOW-2005.  This problem demonstrates the wetting and drying capability in MODFLOW 6. The MODFLOW 6 problem is constructed with two layers (like the MODFLOW-2005 model) but the thickness of the confining bed is included in model layer 2 and the horizontal hydraulic conductivity of layer 2 is half that of model layer 2 in the MODFLOW-2005 model in order to calculate the correct horizontal conductance.
\item ex04-fhb---This problem is included with the MODFLOW-2005 examples. This problem demonstrates how the time-series functionality, combined with the Constant-Head and Well Packages, can be used to replace the Flow and Head Boundary (FHB) Package.
\item ex05-mfusg1disu---This is the first test problem presented in the MODFLOW-USG documentation.  It is included as an example problem to demonstrate a simple unstructured groundwater flow model.  The model uses ghost nodes to improve the accuracy of the groundwater flow solution.
\item ex06-mfusg1disv---This is the first test problem presented in the MODFLOW-USG documentation.  It is included as an example problem to demonstrate the DISV Package for a simple groundwater flow model.  The model uses ghost nodes to improve the accuracy of the groundwater flow solution.
\item ex07-mfusg1lgr---This is also the first test problem presented in the MODFLOW-USG manual; however, it is represented using two separate structured models.  The models are connected using a Groundwater Flow to Groundwater Flow (GWF-GWF) Exchange.  These two models are solved simultaneously in the same matrix equations.  A ghost-node correction is also applied to improve the flow calculation between models.
\item ex08-mfusg1xt3d---This is the first test problem presented in the MODFLOW-USG documentation.  It is included as an example problem to demonstrate the DISV Package for a simple groundwater flow model.  The model uses the XT3D formulation to improve the accuracy of the groundwater flow solution.
\item ex09-bump---This is a one-layer steady-state problem involving wetting and drying.  There is a rise in the bottom surface of the model, and groundwwater flows around the rise.
\item ex10-bumpnr---This is a one-layer steady-state problem designed to test the Newton-Raphson approach.  There is a rise in the bottom surface of the model, and groundwwater flows around the rise.
\item ex11-disvmesh---Demonstration of a triangular mesh with the DISV Package to discretize a circular island with a radius of 1500 meters.  The model has 2 layers and uses 2778 vertices (NVERT) to delineate 5240 cells per layer (NCPL).  General-head boundaries are assigned to model layer 1 for cells outside of a 1025 m radius circle.  Recharge is applied to the top of the model. 

\item ex12-hanicol---Simple steady state model using a regular MODFLOW grid to simulate the response of an anisotropic confined aquifer to a pumping well. A constant-head boundary condition surrounds the active domain.  K22 is set to 100.0, which causes hydraulic conductivity in column direction to be 100 x more than K, which is in row direction.  Drawdown is more pronounced in column direction.

\item ex13-hanirow---Simple steady state model using a regular MODFLOW grid to simulate the response of an anisotropic confined aquifer to a pumping well. A constant-head boundary condition surrounds the active domain.  K22 is set to 0.01, which causes K in column direction to be 100 x less than K in the row direction.  Drawdown is more pronounced in row direction.

\item ex14-hanixt3d---Simple steady state model using a regular MODFLOW grid to simulate the response of an anisotropic confined aquifer to a pumping well. For this problem, the XT3D formulation is used so that hydraulic conductivity ellipse can be rotated in the x-y plane.  A constant-head boundary condition surrounds the active domain.  K22 is set to 0.01, which causes hydraulic conductivity in the column direction (prior to rotation) to be 100 x less than K in the row direction.  This ellipse is then rotated in the x-y plane by specifying a value for ANGLE1 in the NPF Package.  ANGLE1 is specifed with a constant value of 15 degrees for the entire grid, which means the dominant K component is rotated 15 degrees counter clockwise.  Drawdown is more pronounced along the dominant axis of the hydraulic conductivity ellipse.

\item ex15-whirlsxt3d---This is a 10 layer steady-state problem involving anisotropic groundwater flow.  The XT3D formulation is used to represent variable hydraulic conductivitity ellipsoid orientations.  The resulting flow pattern consists of groundwater whirls, as described in the XT3D documentation report.
\item ex16-mfnwt2---This is the the second example problem described in the MODFLOW-NWT documentation (Niswonger and others, 2011) and is based on ``problem 2'' in McDonald and others (1991). A fourth steady-state stress period has been added to the problem for comparison with fig. 8D in Niswonger and others(2011).
\item ex17-mfnwt3h---This is the high recharge case of the third example problem described in the MODFLOW-NWT documentation.
\item ex18-mfnwt3l---This is the low recharge case of the third example problem described in the MODFLOW-NWT documentation.
\item ex19-zaidel---This is the stair-step problem described in Zaidel (2013).  In this simulation, the Newton-Raphson formulation is used to improve simulation convergence.
\item ex20-keating---This is an example problem described in Keating and Zyvoloski (2009).  The problem involves recharge through the unsaturated zone onto an aquitard.  The Newton-Raphson formulation is used for this problem to obtain a solution.
\item ex21-sfr1---This is the stream-aquifer interaction example problem (test 1) from the Streamflow Routing Package documentation (Prudic and others, 1989).  The specified depth segments in the original problem have been converted to active reaches and the diversion has been converted from UPTO to FRACTION CPRIOR type. This problem is simulated using the Streamflow Routing (SFR) Package in MODFLOW 6.
\item ex22-lak2---This is the lake-stream-aquifer interaction example problem (test simulation 2) from the Lake Package documentation (Merritt and Konikow, 2000).  This problem is simulated using the Lake (LAK) and Streamflow Routing (SFR) Packages in MODFLOW 6. The Mover (MVR) Package is also used to exchange water between the SFR and LAK Packages.
\item ex23-lak4---This is the lake-aquifer interaction example problem (test simulation 4) from the Lake Package documentation (Merritt and Konikow, 2000).  This problem is simulated using the Lake (LAK) Package in MODFLOW 6.
\item ex24-neville---This is the multi-aquifer well simulation described in Neville and Tonkin (2004).  This problem is simulated using the Multi-Aquifer Well (MAW) Package in MODFLOW 6.
\item ex25-flowing-maw---This is a multi-aquifer well simulation that demonstrates how to implement the flowing well option available in Multi-Aquifer Well (MAW) Package in MODFLOW 6. Aquifer properties and initial heads are identical to Neville and Tonkin (2004).  The pumping rate for well in the center of the domain is 0.0 cubic meters per day and the flowing well discharge elevation and conductance are specified to be 0.0 meters and 7,500 square meters per day.
\item ex26-Reilly-maw---This is the unstressed multi-aquifer well simulation described in Reilly and others (1989).  This problem is simulated using the Multi-Aquifer Well (MAW) Package in MODFLOW 6.
\item ex27-advpakmvr---This is a variant of the unsaturated zone-lake-stream-aquifer interaction example problem (test simulation 2) from the Unsaturated Zone Flow Package documentation (Niswonger and others, 2006) that includes a two layer aquifer and two lakes connected to the stream network.  This problem is simulated using the Unsaturated Zone Flow (UZF), Lake (LAK), and Streamflow Routing (SFR) Packages in MODFLOW 6. The Mover (MVR) Package is also used to exchange water between the UZF, LAK, and SFR Packages. Infiltration rates, ET rates, streamed Ks and lakebed leakances were changed to lower the water table below the interface of layers 1 and 2. This was done to demonstrate unsaturated flow through multiple layers. Aquifer K values were also changed.
\item ex28-mflgr3---The is Example 3 from the MODFLOW-LGR2 documentation.
\item ex29-vilhelmsen-gc---This is the Globally Coarse (GC) model described in Vilhelmsen et al. (2012).
\item ex30-vilhelmsen-gf---This is the Globally Fine (GF) model described in Vilhelmsen et al. (2012).
\item ex31-vilhelmsen-lgr---This is the Local Grid Refinement (LGR) model described in Vilhelmsen et al. (2012).
\item ex32-periodicbc---Periodic boundary condition problem is based on Laattoe and others (2014). A MODFLOW 6 GWF-GWF Exchange is used to connect the left column with the right column. 
\end{itemize}
